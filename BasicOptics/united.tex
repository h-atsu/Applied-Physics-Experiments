\documentclass[11pt, a4paper]{jsarticle}
\usepackage{multicol}  % パッケージの追加
\usepackage[dvipdfmx]{graphicx}
\usepackage{docmute}

\begin{document}
\begin{titlepage}
  \begin{center}
    {\Huge 基礎光学 \\ Basic Optics}\\ % タイトル
    \vspace{30truept}
    {\huge 提出者 : 08A17153 羽田充宏}\\ % 学籍番号
    \vspace{50truept}

    \begin{list}{}{\setlength{\leftmargin}{80pt}}
    \item {\huge 実験実施日 : 2019年4月8日}\\
    \vspace{10truept}
    \item {\huge 実験実施日 : 2019年4月12日}\\
    \vspace{10truept}
    \item {\huge 実験実施日 : 2019年4月15日}\\
    \vspace{10truept}
    \item {\huge 実験実施日 : 2019年4月19日}\\
    \vspace{10truept}
    \item {\huge 実験実施日 : 2019年4月22日}\\
    \vspace{10truept}
    \item {\huge 実験実施日 : 2019年4月26日}\\
    \end{list}

    \vspace{50truept}
    {\huge 提出日 : 5月13日}\\ % 提出日
    \vspace{30truept}
    \begin{list}{}{\setlength{\leftmargin}{70pt}}
    \item {\huge 共同実験者 : 長野 貴裕}
    \item {\huge      西尾 晃}
    \item {\huge      畠中 駿平}
    \item {\huge      能登 健太}
    \end{list}
  \end{center}
\end{titlepage}
%========================================================
\section*{Porpose and abstract of experimental theme}
この実験では光学の基礎知識と研究での実践的な実験を学ぶ.実験は”光学装置の扱い”,”ビームの拡大”,”光の回折”,”光の干渉”,”光の偏光”,”ホログラム”
といった光学の幅広い題目を扱っていく.
% =======================================================
% =======================================================
\documentclass[11pt, a4paper,twocolumn]{jarticle}
\usepackage[dvipdfmx]{graphicx}
\begin{document}
%=============================================================
\section{Characteristics of \\NOT gates ($1^{st} day$)}
\subsection{Purpose}
異なるICによる論理ゲートの挙動を実験する.
また与えられた電気回路を計画的に製作し,NOTゲートの入出力電圧を計測する.
\subsection{Equipment}
\begin{itemize}
    \item NOT gate IC(74LS04,74HC04,74HC14)
    \item Universal printed circuit board ICB-86
    \item Switching power supply with a connector
    \item Connector for power suuply
    \item Trinmming potentiometer 1k$\Omega$
    \item 14pin IC socket
    \item 8 pin IC socket
    \item Ceramic capacitor 0.1$\mu$
    \item Dogital multimeters
    \item Clip test leads
\end{itemize}
\subsection{Procedure}
今回は半田付けによって論理回路を実装していく.
NOTゲート回路を図\ref{fig:5}のように作る.
まず,スイッチング電源からの5V,GNDは8ピンソケットを使って基盤中央の縦のラインに給電する.
可変抵抗を基盤に取り付け,ジャンパー線と半田によって回路を作っていく.
今回は一つのNOTゲートのみ使えれば良いので1,2番ピンのゲートを使えるように14ピンICソケットを配線する.
また使わないゲートの入力端子はGNDまたは5Vに接続し,出力端子は解放する.
次に電源と接続し可変抵により入力電圧が0~5Vの間で変化するか確認する.
確認できたら電源を外し,周辺回路への電源電圧の変動の影響や磁気ノイズを減らすために電源とGNDの間にセラミックコンデンサー(0.1$\mu$)を接続する.
回路が完成したらそれぞれのICをソケットに取り付けテスターを接続し入力電源と出力電源を測定する.
\begin{figure}[htbp]
 \begin{center}
  \includegraphics[width=0.8\linewidth]{fig5.png}
 \end{center}
 \caption{NOTゲート回路}
 \label{fig:5}
\end{figure}

\subsection{Result}
図\ref{fig:1},図\ref{fig:2},図\ref{fig:3},図\ref{fig:4}は実験より得られた$V_{in}$,$V_{out}$の関係をグラフにしたものである.
全てのICにおいて入力電圧($V_{in}$)が5V付近では出力電圧($V_{out}$)は0Vを示し,入力電圧($V_{in}$)が0V付近では出力電圧($V_{out}$)は5Vを示を示した.
またICによって$V_{in}$を下げていった際の$V_{out}$の電圧の上がり方の鋭さが違っていた.74LS04,74HC04,74HC14の順で出力電圧の上がり方は鋭くなっていった.
また74HC14に関しては$V_{in}$を上げていく場合と下げていく場合で異なる動作をした.
$V_{in}$を下げていった場合では2Vを境として$V_{out}$が5V付近を示したが,$V_{in}$を上げていった場合は3Vを境として$V_{out}$が0Vとなった.
\begin{figure}[htbp]
 \begin{center}
  \includegraphics[width=0.8\linewidth]{fig1.png}
 \end{center}
 \caption{74LS04}
 \label{fig:1}
\end{figure}

\begin{figure}[htbp]
 \begin{center}
  \includegraphics[width=0.7\linewidth]{fig2.png}
 \end{center}
 \caption{74HC04}
 \label{fig:2}
\end{figure}


\begin{figure}[htbp]
 \begin{center}
  \includegraphics[width=0.7\linewidth]{fig3.png}
 \end{center}
 \caption{74HC14($V_{in}$下げていく)}
 \label{fig:3}
\end{figure}

\begin{figure}[htbp]
 \begin{center}
  \includegraphics[width=0.7\linewidth]{fig4.png}
 \end{center}
 \caption{74HC14($V_{in}$上げていく)}
 \label{fig:4}
\end{figure}

\subsection{Discussion}
今回はどのICについても入力電圧が5V付近の時出力電圧は0V付近を示し,逆に入力電圧が0付近の時出力電圧は5V付近を示したのでNOTゲートが正しく機能したと推測できる.

またTTLにおけるNOTゲートの動作を考察する.
\begin{figure}[htbp]
 \begin{center}
  \includegraphics[width=0.7\linewidth]{fig6.png}
 \end{center}
 \caption{TTL NOTgate}
 \label{fig:6}
\end{figure}

TTLにおけるNOTゲートは図\ref{fig:6}のように表せる.
この回路において入力電圧$V_{in}$が0Vの時トランジスタQ1にいおいてベースエミッタ電圧$V{BE}$はスイッチング電圧である0.6Vを超えるためQ1は作動する.
この時Q2において$V_{BE}$は0VとなるためQ2は作動せず抵抗による電圧降下が起こらないため出力電が5Vとなると考えられる.

一方,入力電圧$V_{in}$が5V付近の時Q1において$V_{BE}$は逆バイアスとなりベースエミッタ間には電流は流れなくなる.
この時ベースコレクタ間に電流が流れることとなりトランジスタQ2は作動し抵抗により電圧降下を起こすこととなる.
したがって$V_{out}$は0Vを示すと考えられる.

次にCMOSにおけるNOTゲートの動作を考察する.
\begin{figure}[htbp]
 \begin{center}
  \includegraphics[width=0.7\linewidth]{fig7.png}
 \end{center}
 \caption{TTL NOTgate}
 \label{fig:7}
\end{figure}

図\ref{fig:7}のCOMOS回路ではpチャネル型(Q1)とnチャネル型(Q2)の二つのMOSFETを組み合わせて作られている.
それぞれ入力電圧$V_{in}$が0V付近の時はQ1がオンとなり出力電圧$V_{out}$は5Vとなり,入力電圧$V_{in}$が5V付近の時Q2がオンとなり出力電圧$V_{out}$は0V付近になることが予想される.

以上の考察から74LS04を用いた回路が他の三つのICを用いた回路よりも入出力電圧のグラフが緩やかであったのはトランジスタにおいてスイッチング電圧を境目として完全に電流が流れる流れないの関係が成り立つのではなく,スイッチング電圧の付近で少しずつ電流が流れ始めるためであると予想できる.

またCMOS回路においては回路内部で入力電圧によって選択的にスイッチのようにon,offのように作動するのでトランジスタを用いたTTL回路に比べて鋭い入出力電圧のグラフが得られたと考えられる.

% ヒステリシスの考察





%=============================================================
\newpage
\end{document}

\documentclass[11pt, a4paper]{jsarticle}
\usepackage{multicol}  % パッケージの追加
\usepackage[dvipdfmx]{graphicx}
\begin{document}
%=============================================================
%=============================================================
\section{Diffraction from circular apertures and slits}
\subsection*{Purpose}
実験の目的は円形開口やスリットの干渉模様の観察し,干渉から開口の直径,スリット幅,ダブルスリットの間隔を計算することである.
\subsection{Circular aperture}
\subsubsection{Procedure}
図\ref{fig:five}に示すように光学系を組み立てる.
またそれぞれの距離を図\ref{fig:five}に示すように名前をつける.
まず,レーザーの前に円形開口を置きそこから離れた距離にスクリーンを設置する.
今回は$02mm$,$0.4mm$の二つの円形開口を用いて実験を行う.
また,スクリーンは方眼紙で製作する.
次に開口からクリーンまでの距離$L$を測定する.
レーザーの電源をつけるとスクリーンに円形の干渉縞を得る.
次に円形の干渉縞の中心から1番目の暗線のまでの距離$r$を測定する.
これを二種類の円形開口に対して距離$L$を変えながら二回ずつ計測を行う.
\begin{figure}[htbp]
 \begin{center}
  \includegraphics[width=100mm]{fig5.png}
 \end{center}
 \caption{円形開口の光学系}
 \label{fig:five}
\end{figure}\\

またこれらの測定した距離は式(\ref{eq:b})の関係をみたす.
\begin{equation}
    sin\theta = \frac{1.22\lambda}{D} \label{eq:b}
\end{equation}\\


またこの時$\theta$,$D$の値は非常に小さいので$sin\theta \simeq tan\theta = r/L$とみなすことができる.
また今回使用したHe-Ne Laserの波長は$\lambda = 632.8nm$である.
以上より計測値を関係式へ代入して円形開口の直径を推測する.
その後0.2mm円形開口を光学顕微鏡によって計測しその直径を実際に求めた.

\subsubsection{Result}
測定結果から次の表が得られた.
\begin{table}[htb]
 \begin{minipage}{0.45\hsize}
  \begin{center}
    \caption{$0.2mm$円形開口}
    \begin{tabular}{rrr} \hline
        $r(mm)$ & $L(mm)$ & $D(mm)$  \\ \hline
        0.25     & 583 & 0.18\\
        0.45    & 1435 & 0.246\\ \hline
    \end{tabular}
    \label{tab:b}
  \end{center}
 \end{minipage}
 \begin{minipage}{0.45\hsize}
  \begin{center}
    \caption{$0.4mm$円形開口}
    \begin{tabular}{rrr} \hline
        $r(mm)$ & $L(mm)$ & $D(mm)$  \\ \hline
        2.6   & 974 & 0.289\\
        3.5    & 1430 & 0.315\\ \hline
    \end{tabular}
    \label{tab:c}
  \end{center}
 \end{minipage}
\end{table} 


またそれぞれ以下のような干渉模様が観測できた.
\begin{figure}[htbp]
 \begin{minipage}{0.45\hsize}
  \begin{center}
   \includegraphics[width=60mm]{fig6.png}
  \end{center}
  \caption{$0.2mm$円形開口の干渉縞}
  \label{fig:six}
 \end{minipage}
 \begin{minipage}{0.45\hsize}
  \begin{center}
   \includegraphics[width=60mm]{fig7.png}
  \end{center}
  \caption{$0.4mm$円形開口の干渉縞}
  \label{fig:seven}
 \end{minipage}
\end{figure}

また0.2mm円形開口の実測値は0.19mmであった.
干渉模様は円の半径が大きくなっていくに連れて明暗の境目が不明瞭になって区別がつかなくなった.

\newpage
また以下は光学顕微鏡で観測した0.2mm円形開口の写真である.
図\ref{fig:27}における一目盛りは10${\mu}m$であるのでそれぞれの画像を透過させて円形開口の大きさを測定した.
その結果円形開口は0.19mmであることが測定された.

\begin{figure}[htbp]
 \begin{minipage}{0.45\hsize}
  \begin{center}
   \includegraphics[width=60mm]{fig26.png}
  \end{center}
  \caption{光学顕微鏡で観察した$0.2mm$円形開口}
  \label{fig:26}
 \end{minipage}
 \begin{minipage}{0.45\hsize}
  \begin{center}
   \includegraphics[width=60mm]{fig27.png}
  \end{center}
  \caption{光学顕微鏡で計測した目盛り}
  \label{fig:27}
 \end{minipage}
\end{figure}
\subsubsection{Discussion}
$0.2mm$円形開口の実験値と実測値である0.19mmとの差は比較的小さかった.一方で$0.4mm$円形開口の理論値との差はかなりの開きがあった.
また円形開口のプレートは複数個の開口があったためにそれぞれの円形が偏っていたりしたことも考えられる.
さらに他の要因としてそもそも$D$を算出する際に$sin\theta \simeq tan\theta = r/L$と近似したために実際の値とは異なる結果となった可能性が考えられる.
さらには目測による測定に誤差があったなどの要因も考えられる.
また干渉には開口部の大きさがスクリーンまでの距離に対して十分小さい時に起こるフレネル回折,ビーム源もしくは観測点がビームを回折するものから無限遠に位置する時に起こるフラウンホーファー回折などがあるが今回の実験では観測点はビームを回折する開口に比べて十分大きいと考えられるのでフラウンホーファー回折であると考えられる.
%=============================================================
\subsection{Single Slit}
\subsubsection{Procedure}
基本的には前回と同じ光学系を組み立てる.
今回は円形開口の代わりにシングルスリットをレーザーの前に置く.
以下の図はシングルスリットの光学系である.
今回は$r$,$L$の距離を測定する事によって$\omega$の値を推測する.
またこの実験では$0.1mm$,$0.2mm$二つのスリットを用いてそれぞれ1回ずつ測定を行う.
\begin{figure}[htbp]
 \begin{center}
  \includegraphics[width=100mm]{fig8.png}
 \end{center}
 \caption{シングルスリットの光学系}
 \label{fig:eight}
\end{figure}\\

またこれらの測定した距離は
\begin{equation}
    sin\theta = \frac{\lambda}{\omega} \label{eq:c}
\end{equation}\\
式(\ref{eq:c})の関係をみたす.
またこの時$\theta$,$D$の値は非常に小さいので$sin\theta \simeq tan\theta = r/L$とみなすことができる.
また今回使用したHe-Ne Laserの波長は$\lambda = 632.8nm$である.
以上より計測値を関係式へ代入してシングルスリットの間隔を推測する.
今回は時間の関係上実際に光学顕微鏡を用いてスリット幅を測定することはできなかった.

\subsubsection{Result}
測定結果から次の表が得られた.
また式(\ref{eq:c})の関係よりスリット幅が大きくなるにつれて干渉縞の間隔は小さくなっていくことが予想される.実際に実験からもこの関係が確かめられた.
\begin{table}[htb]
 \begin{minipage}{0.45\hsize}
  \begin{center}
    \caption{$0.1mm$シングルスリット}
    \begin{tabular}{rrr} \hline
        $r(mm)$ & $L(mm)$ & $\omega(mm)$  \\ \hline
        7.0    & 974 & 0.088\\ \hline
    \end{tabular}
    \label{tab:d}
  \end{center}
 \end{minipage}
 \begin{minipage}{0.45\hsize}
  \begin{center}
    \caption{$0.2mm$シングルスリット}
    \begin{tabular}{rrr} \hline
        $r(mm)$ & $L(mm)$ & $\omega(mm)$  \\ \hline
        3.5    & 974 & 0.176\\ \hline
    \end{tabular}
    \label{tab:e}
  \end{center}
 \end{minipage}
\end{table}

またそれぞれ以下のような干渉模様が観測できた.
\begin{figure}[htbp]
 \begin{minipage}{0.45\hsize}
  \begin{center}
   \includegraphics[width=60mm]{fig9.png}
  \end{center}
  \caption{$0.1mm$シングルスリットの干渉縞}
  \label{fig:nine}
 \end{minipage}
 \begin{minipage}{0.45\hsize}
  \begin{center}
   \includegraphics[width=60mm]{fig10.png}
  \end{center}
  \caption{$0.2mm$シングルスリットの干渉縞}
  \label{fig:ten}
 \end{minipage}
\end{figure}
\subsubsection{Discussion}
いずれの結果も理論値よりも$0.02mm$ほど小さくなってしまったのでどちらも目測による計測の時実際の感覚よりも大きく読んでしまった可能性が考えられる.
また光学顕微鏡を用いて実測値を計算していないので実際にスリット幅が0.2mm,0.1mmではなかったために誤差が生じていることも考えられる.
%=============================================================
\subsection{Yong Double Slit}
\subsubsection{Procedure}
光学系を図\ref{fig:eleven}に示すように製作する.
ダブルスリットをビームの経路に置きビーム光を回折させスクリーンに干渉縞を映し観察する.

\begin{figure}[htbp]
 \begin{center}
  \includegraphics[width=100mm]{fig11.png}
 \end{center}
 \caption{ダブルスリットの光学系}
 \label{fig:eleven}
\end{figure}

またこの時以下の式(\ref{eq:d}),(\ref{eq:f})の関係が成り立つ.それぞれ測定した$\Delta \theta$,$\Delta x$を式に代入することで二つのスリット間隔$d$
を推測する.
また今回は$0.2mm$,$0.1mm$の二つのスリット間隔をを持つダブルスリットに対して一回ずつ計測を行う.
また今回も時間の都合上ダブルスリットの間隔を光学顕微鏡を用いて計りはしなかった.

\begin{equation}
    \Delta\theta = \frac{\Delta x}{R} \label{eq:d}
\end{equation}
\begin{equation}
    \Delta\theta = \frac{\lambda}{d} \label{eq:f}
\end{equation}

\subsubsection{Result}
測定結果から次の表が得られた.

\begin{table}[htb]
 \begin{minipage}{0.45\hsize}
  \begin{center}
    \caption{$0.1mm$ダブルスリット}
    \begin{tabular}{rrr} \hline
        $\Delta x(mm)$ & $R(mm)$ & $d(mm)$  \\ \hline
        7.0    & 974 & 0.088\\ \hline
    \end{tabular}
    \label{tab:f}
  \end{center}
 \end{minipage}
 \begin{minipage}{0.45\hsize}
  \begin{center}
    \caption{$0.2mm$ダブルスリット}
    \begin{tabular}{rrr} \hline
        $\Delta x(mm)$ & $R(mm)$ & $d(mm)$  \\ \hline
        2.8    & 974 & 0.22\\ \hline
    \end{tabular}
    \label{tab:g}
  \end{center}
 \end{minipage}
\end{table}

また以下のような干渉縞が観察された.

\begin{figure}[htbp]
 \begin{minipage}{0.45\hsize}
  \begin{center}
   \includegraphics[width=60mm]{fig12.png}
  \end{center}
  \caption{$0.1m$ダブルスリットの干渉縞}
  \label{fig:12}
 \end{minipage}
 \begin{minipage}{0.45\hsize}
  \begin{center}
   \includegraphics[width=60mm]{fig13.png}
  \end{center}
  \caption{$0.2m$ダブルスリットの干渉縞}
  \label{fig:13}
 \end{minipage}
\end{figure}

\subsubsection{Discussion}
どちらの誤差も$-0.02mm$であったので目測の際に$r$を大きく読んでしまったことなどが考えられる.
またシングルスリットの時と同じような干渉縞が得られたがシングルスリットの方が中心が明るく,ダブルスリットの時は中心から離れた干渉縞でも強度が高かったことが観測された.これはダブルスリットの際は二つのスリットを通過した際にビーム光が素元波となり互いの波が干渉をするためだと考えられる.

%==========================================================
%==========================================================
\newpage
\end{document}

\documentclass[11pt, a4paper]{jsarticle}
\usepackage{multicol}  % パッケージの追加
\usepackage[dvipdfmx]{graphicx}
\begin{document}
%=============================================================
%=============================================================
\section{Michelson Interferometer and Coherence}
\subsection*{Purpose}
この実験の目的はマイケルソン干渉系によって作られる干渉模様を観測することと,実験で使用しているHe-Neレーザーのコヒーレント長をマイケルソン干渉系を用いて測定することである.
\subsection{Michelson Interferometer}
\subsubsection{Procedure}
まず以下のように干渉系を組み立てる.
\begin{figure}[htbp]
 \begin{center}
  \includegraphics[width=100mm]{fig14.png}
 \end{center}
 \caption{マイケルソン干渉系}
 \label{fig:14}
\end{figure}\\

まず光の強度を弱めるためにNDフィルター(25\%)をレーザーの前に設置する.
次にビームエキスパンダーを組み立てる.
今回はガリレオ型のエキスパンダーを組み立てる.
また,それぞれのミラーの高さを合わせて跳ね返った光がスクリーン上で一致するように調整を行う.
次にM2のミラーを移動させることで干渉縞が観測されたので干渉模様を写真に収める.
さらにその状態から机を叩く事により干渉模様がどのように変化するかを観察する.
その後エキスパンダーの二個目の凸レンズを移動させて二つのレンズ間距離を小さくする事で干渉模様にどのような変化が起きたかを観察する.
得られた干渉縞の写真をカメラに収める.
またこの時$BS-M1 = 8.0cm$,$BS-M2 = 5.0cm$の距離であった.

\subsubsection{Result}
二つの光がスクリーン上で重なり干渉を起こす事で図\ref{fig:15}のような干渉模様が得られた.
また机を叩いて揺れを起こす事でスクリーン上に映っていた干渉模様が消えた.
その後0.98[s]後に元どおりの干渉模様をまた形成した.
さらにエキスパンダーのレンズ間距離を小さくすると干渉模様が図\ref{fig:16}のように円の一部分のように丸くなって映った.
さらにレンズ間距離を元の位置よりも大きくすると同じような円状の干渉模様が得られた.

\begin{figure}[htbp]
 \begin{minipage}{0.45\hsize}
  \begin{center}
   \includegraphics[width=60mm]{fig15.png}
  \end{center}
  \caption{干渉模様}
  \label{fig:15}
 \end{minipage}
 \begin{minipage}{0.45\hsize}
  \begin{center}
   \includegraphics[width=60mm]{fig16.png}
  \end{center}
  \caption{レンズ間を狭くした時の干渉模様}
  \label{fig:16}
 \end{minipage}
\end{figure}

\subsubsection{Discussion}
まずM2の位置を調節する事でスクリーン上に平行な干渉模様が得られたからM1に反射された光とM2に反射された光がそれぞれ平面波であってその波がスクリーン上でぶつかる事によって干渉模様が得られたと考えられる.
この事実から実験で使用したレーザーはコヒーレントな光であると言える.

机を叩いた際に干渉模様が見えなくなったのは叩いた振動によって一時的にそれぞれの光路長が異なりスクリーン上で干渉条件が満たされなくなったためだと考えられる.

またミラーの角度を変えるとスクリーン上で観測される干渉模様の縞の感覚が異なったのは二つのビーム光の入社角度が異なるためだと考えられる.

さらにエキスパンダーのレンズ間距離を縮めた際に同心円状の干渉模様が得られたのは平行光だったビームエキスパンダーによって発散してビーム光が球面波になった為だと考えられる.

またマイケルソン干渉系は二つのビーム光の経路差に強く依存するため光路の途中にガラスなどを挿入することでガラスの屈折率を測定することに応用ができると予測される.

%=============================================================
\subsection{Coherence}
\subsubsection{Procedure}
前回の実験と同様に図\ref{fig:14}の光学系を製作する.初めの$BS-M1 = 8.0cm$,$BS-M2 = 5.0cm$の状態から$BS-M1$の距離のみを次第に広げていきスクリーン状で干渉模様が得られなくなるまで広げていきコヒーレント長を決定する.
\subsubsection{Result}
$BS-M1 = 10.0cm$の時は干渉模様が観測された.しかし$BS-M1 = 12.5cm$にすると干渉模様は見れなくなった.
以上の結果より可干渉光路差は$(|BS-M1|-|BS-M2|) \times 2 = 4.0cm$である事がわかった.
\subsubsection{Discussion}
一般に実験用のビーム光のコヒーレント長は数十cmであるので今回の実験から得られた値とは大きな開きがあった.
この原因としては二つの光の交差する角度が小さかったために干渉模様が得られなかった,調節不足で本当はあったのに観測できなかったなどの理由が考えられる.

%==========================================================
%==========================================================
\newpage
\end{document}

\documentclass[11pt, a4paper,twocolumn]{jarticle}
\usepackage[dvipdfmx]{graphicx}
\usepackage{listings,jlisting}

\begin{document}
%=============================================================
\section{Temprature dependence of the resistance($4^{th} day$)}

\subsection{Purpose}
金属抵抗の温度による変化を学ぶ.
\subsection{Procedure}
まず試料を四端子測定法の測定回路に接続する.次に熱電対と共に試料を固定し液体窒素の入った容器の中に沈めていく.
この時試料の温度は熱電対の起電力によって測定する.
今回使用した熱電対はクロメルーアルメル熱電対であり測定部分を氷水の容器に浸し基準を273Kにして測定する.
測定器の製作は図\ref{fig:29}のように組み立てる.

以上の手順でCu,NiCr,Wの低効率を測定していく.
測定は室温から77K程度まで5段階に分けて測定を行なった.
測定結果より抵抗値を最小二乗法で求めその温度依存性をグラフに示す.

\begin{figure}[htbp]
 \begin{center}
  \includegraphics[width=0.8\linewidth]{fig29.png}
 \end{center}
 \caption{温度依存性の実験装置}
 \label{fig:29}
\end{figure}

\subsection{Result}
測定の結果温度と抵抗値の関係をプロットすると以下のようなグラフが得られた.

\begin{figure}[htbp]
 \begin{center}
  \includegraphics[width=0.8\linewidth]{fig30.png}
 \end{center}
 \caption{Wの温度依存}
 \label{fig:30}
\end{figure}

\begin{figure}[htbp]
 \begin{center}
  \includegraphics[width=0.8\linewidth]{fig31.png}
 \end{center}
 \caption{NiCrの温度依存}
 \label{fig:31}
\end{figure}

\begin{figure}[htbp]
 \begin{center}
  \includegraphics[width=0.8\linewidth]{fig32.png}
 \end{center}
 \caption{Cuの温度依存}
 \label{fig:32}
\end{figure}

\newpage


\subsection{Discussion}
実験結果より温度が低くなるにつれて抵抗率が小さくなった原因について考える.
まず前回の考察より金属中において電流の流れを阻害するものは熱振動する格子だと考えられることがわかった.したがって温度が下がるとその分熱運動により振動する格子の振動の激しさは穏やかになることが予想され結果として電流が格子と衝突する回数がすくなり流れる電流量が多くなり抵抗率が小さくなることが予想される.

%=============================================================
\newpage
\end{document}

\documentclass[11pt, a4paper,twocolumn]{jarticle}
\usepackage[dvipdfmx]{graphicx}
\usepackage{amssymb}
\begin{document}
%=============================================================
\section{溶媒粘性と楕円球にかかる放射圧の測定 (5日目)}
\subsection{実験目的}
今回の実験目的はポリスチレン球のトラッピングによりエタノールの粘性を計算し,その値を理科年表と比較することと楕円型のポリスチレン球を用いたトラッピングを行い球形とどのような違いがあるのかを観察することである.
\subsection{実験手順}
ポリスチレン球をエタノールに分散させ,前回と同じくトラッピングできる最大速度を求めたのち4日目の結果を参考にして式\ref{eq:3}よりエタノールの粘性を求めた.
この時トラップ力Fを図\ref{fig:4}の直線近似から求め代入した.

次に楕円型のポリスチレン球を純水に分散させて同様にトラッピングできる最大速度を求める.この際に楕円粒子のアスペクト比によるトラッピング性能の違いを観察する.

\begin{equation}
    \eta = \frac{F}{3\pi av}
\label{eq:3}
\end{equation}

\subsection{結果}
それぞれの大きさにおける測定結果とエタノール粘性推定値は表\ref{table:5},表\ref{table:6},表\ref{table:7}のようになった.
室温におけるエタノールの粘性理想値は$1.08 \times{10^{-3}}$であるから全体的に理想値よりも大きい値を出した.

また楕円粒子のトラッピングは出力電流が0.8Aから1.0Aの時捕捉でき,それ以外の電流の時はトラップ力が強すぎたり弱すぎたりしたためにトラップすることができなかった.
またモニターで確認した際のアスペクト比が1.36:1の楕円粒子はトラッピングに成功した一方で2:1の楕円粒子のトラッピングはできなかった.
またどの楕円粒子もトラッピングした直後にモニター上で円形に変形した.

\begin{table}[htbp]
    \begin{center}
        \scalebox{0.8}[0.9]{ %ココ
        \begin{tabular}{cccc}
            光強度[mW] & FPPS & トラップ力[N] & エタノール粘性 \\ \hline
            19.56  & 13000 & $4.111\times 10^{-12}$  & $1.678\times 10^{-3}$  \\
            49.72 & 32000 & $1.0147\times 10^{-11}$  & $1.681\times 10^{-3}$ \\
            81.45 & 50000 & $1.649\times 10^{-11}$  & $1.750\times 10^{-3}$
        \end{tabular}
        }
        \caption{5$\mu m$の粘性測定}
        \label{table:5}
    \end{center}
\end{table}

\begin{table}[htbp]
    \begin{center}
        \scalebox{0.8}[0.9]{ %ココ
        \begin{tabular}{cccc}
            光強度[mW] & FPPS & トラップ力[N] & エタノール粘性 \\ \hline
            19.741624  & 8000 & $5.948\times 10^{-12}$  & $1.972\times 10^{-3}$  \\
            49.847876 & 23000 & $1.197\times 10^{-11}$  & $1.380\times 10^{-3}$ \\
            86.42324 & 36000 & $1.928\times 10^{-11}$  & $1.421\times 10^{-3}$
        \end{tabular}
        }
        \caption{10$\mu m$の粘性測定}
        \label{table:6}
    \end{center}
\end{table}

\begin{table}[htbp]
    \begin{center}
        \scalebox{0.8}[0.9]{ %ココ
        \begin{tabular}{cccc}
            光強度[mW] & FPPS & トラップ力[N] & エタノール粘性 \\ \hline
            20.48806  & 27000 & $2.749\times 10^{-12}$  & $1.350\times 10^{-3}$  \\
            50.034485 & 74000 & $5.703\times 10^{-12}$  & $1.022\times 10^{-3}$ \\
            92.332525 & 101000 & $9.933\times 10^{-12}$  & $1.304\times 10^{-3}$
        \end{tabular}
        }
        \caption{2$\mu m$の粘性測定}
        \label{table:7}
    \end{center}
\end{table}

\subsection{考察}
エタノールの粘度が全体的に大きな値を出したことについてはサンプルポリスチレン球の粒径がそれぞれ異なっていたことやトラッピング成功の判定基準が曖昧であったために実際の最大速度よりも小さな値を採用してしまったことなどが考えられる.
より正確な粘度を推定するためには粒径の大きさの精度を高くする,トラッピング判定を複数回行う,検量線を作る際のサンプル数を増やすなどが考えられる.
また今回の実験では溶媒がエタノールだったので揮発による水面の揺れによるノイズが加わった可能性が考えられる.
そのためガラスボトムディッシュよりも密閉性の高い容器に入れて測定する方法などが考えられる.

次に楕円粒子のトラッピングの際にモニター上で楕円粒子が円になったのは楕円内をレーザーが通過した際に対称性が最も高くなるのが楕円粒子が直立した時であるためだと考えられる.
そのため横向きの楕円粒子についてもレーザー光が入射した瞬間に直立するような方向のモーメントが働きトラップ状態では常に直立状態あると考えられる.
またアスペクト比が大きくなるにつれてトラッピングが難しくなった理由はアスペクト比の増加に伴い球に比べて力が等方的に加わらなくなり不安定になるために台の揺れやトラッピングの移動によってトラッピング状態がすぐ解除されるためだと考えられる.

%=============================================================
\newpage
\end{document}

\documentclass[11pt, a4paper,twocolumn]{jarticle}
\usepackage[dvipdfmx]{graphicx}
\usepackage{listings,jlisting}

\begin{document}
%=============================================================
\section{Measurement of the temperature dependence of the resistivity of $Bi_2Sr_2Ca_2Cu_3O_{10}$ as a superconductor($6^{th} day$)}

\subsection{Purpose}
超電導の転移温度の測定を通して超電導の原理について学ぶ.
\subsection{Procedure}
前回同様に図\ref{fig:29}のように温度依存性の測定のための実験装置を組み立てる.
次に超電導リボン($Bi_2Sr_2Ca_2Cu_3O_{10}$)の電圧-電流特性を四端子測定法で測定する.
この時超電導のヒステリシス特性を観察するために室温から超電導が起こる相転移温度まで下げていった場合と,超電導状態から室温に近づけていった場合の二パターンについて測定を行いそれぞれの結果を温度-抵抗値のグラフにプロットしていく.
測定の際は電流が100mAを超えないように注意して行う.

\subsection{Result}
測定の結果温度と抵抗値の関係をプロットすると以下のようなグラフが得られた.
ヒステリシスが測定され,室温から温度を下げていった場合よりも超電導状態から室温にあげていった時は結果が全体的に左にシフトした.
また実験結果より温度下降時は相転移温度は140Kほど,上昇時は130Kほどであることが確認された.

\begin{figure}[htbp]
 \begin{center}
  \includegraphics[width=0.8\linewidth]{fig39.png}
 \end{center}
 \caption{温度下降}
 \label{fig:39}
\end{figure}

\subsection{Discussion}
超電導が起こる理由について考える.
まず超電導が起きる極低温状態においては原子の熱振動は十分小さくなっていることが予想される.そこで一つの電子が原子に衝突すると原子が動かされ,プラスに帯電している原子は別の電子を引き寄せるという原理によって結果として電子同士が引き寄せられる現象が図\ref{fig:40}が起きる.
以上のようにして電子のペアについて片方が原子に衝突しても,その衝撃を片方がプラス,もう一方がマイナスに分散吸収してペア全体としては何事もなかったように進むことによって抵抗を感じずに電流を流すことができる.

次に超電導の応用について考える.
超電導状態では小さい電圧で多くの電流を流すことができるので非常に強い磁場を作り出すことが可能である.これはリニアモーターカーなどで利用される.
また量子コンピュータにおいては量子状態がノイズによって壊れることが問題であるためノイズを小さくするために超電導状態が用いられる.
もし超電導状態を室温で実現することができればリニアモータカーが新幹線よりもコストの低い乗り物になることができ,量子コンピュータの製作コストが格段に低くなることが予想される.

\begin{figure}[htbp]
 \begin{center}
  \includegraphics[width=0.8\linewidth]{fig40.png}
 \end{center}
 \caption{超電導メカニズム}
 \label{fig:40}
\end{figure}
%=============================================================
\newpage
\end{document}

\end{document}
