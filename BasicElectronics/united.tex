\documentclass[11pt, a4paper,twocolumn]{jarticle}
\usepackage[dvipdfmx]{graphicx}
\usepackage{docmute}
\begin{document}
\begin{titlepage}
  \begin{center}
    {\Huge 基礎エレクトロニクス}\\
    \vspace{10truept}
    {\Huge Basic Electoronics}\\
    \vspace{30truept}
    {\huge 提出者 : 08A17153 羽田充宏}\\ % 学籍番号
    \vspace{50truept}

    \begin{list}{}{\setlength{\leftmargin}{95pt}}
    \item {\huge 実験実施日 : 2019年5月13日}\\
    \vspace{10truept}
    \item {\huge 実験実施日 : 2019年5月17日}\\
    \vspace{10truept}
    \item {\huge 実験実施日 : 2019年5月20日}\\
    \vspace{10truept}
    \item {\huge 実験実施日 : 2019年5月24日}\\
    \vspace{10truept}
    \item {\huge 実験実施日 : 2019年5月27日}\\
    \vspace{10truept}
    \item {\huge 実験実施日 : 2019年5月31日}\\
    \vspace{40truept}

    \end{list}
    \vspace{50truept}

  \end{center}
\end{titlepage}


% template======================================================
% \section{}
% \subsection{}
% \subsubsection{Purpose}
% \subsection{Equipment}
% \begin{itemize}
%     \item
% \end{itemize}
% \subsubsection{Procedure}
% \subsubsection{Result}
% \subsubsection{Discussion}
%=============================================================
% \begin{figure}[ht]
%  \begin{center}
%   \includegraphics[width=0.8\linewidth]{fig1.png}
%  \end{center}
%  \caption{74HC14}
%  \label{fig:1}
% \end{figure}
% \clearpage
%=============================================================
\twocolumn
[
\begin{center}
    \textbf{\Large Introduction}\\
    \begin{flushleft}
        この実験では電気回路の取り扱いや考え方ついて学ぶ。
        電気回路を応用物理学実験として学ぶ理由には以下のことが挙げられる。
    \end{flushleft}
    \begin{itemize}
        \item 物理学実験において電気回路は重要な役割を果たし,システムの安定性判定や信号増幅の際に必要となるため.
        \item 電気回路を通して様々な物理法則に触れることができる.
        \item 電気回路の中に用いられている半導体は物性物理学のもっとも偉大な発明の一つであり,電気回路を製作する中でどのように半導体が働くかを確認することができる.
        \item デジタル回路はコンピューターアーキテクチャの根幹であるので回路を理解するととはコンピュータテクノロジーを理解することに通じるため.
    \end{itemize}
\end{center}
\vspace{30truept}
]
%=============================================================
\documentclass[11pt, a4paper,twocolumn]{jarticle}
\usepackage[dvipdfmx]{graphicx}
\begin{document}
%=============================================================
\section{Characteristics of \\NOT gates ($1^{st} day$)}
\subsection{Purpose}
異なるICによる論理ゲートの挙動を実験する.
また与えられた電気回路を計画的に製作し,NOTゲートの入出力電圧を計測する.
\subsection{Equipment}
\begin{itemize}
    \item NOT gate IC(74LS04,74HC04,74HC14)
    \item Universal printed circuit board ICB-86
    \item Switching power supply with a connector
    \item Connector for power suuply
    \item Trinmming potentiometer 1k$\Omega$
    \item 14pin IC socket
    \item 8 pin IC socket
    \item Ceramic capacitor 0.1$\mu$
    \item Dogital multimeters
    \item Clip test leads
\end{itemize}
\subsection{Procedure}
今回は半田付けによって論理回路を実装していく.
NOTゲート回路を図\ref{fig:5}のように作る.
まず,スイッチング電源からの5V,GNDは8ピンソケットを使って基盤中央の縦のラインに給電する.
可変抵抗を基盤に取り付け,ジャンパー線と半田によって回路を作っていく.
今回は一つのNOTゲートのみ使えれば良いので1,2番ピンのゲートを使えるように14ピンICソケットを配線する.
また使わないゲートの入力端子はGNDまたは5Vに接続し,出力端子は解放する.
次に電源と接続し可変抵により入力電圧が0~5Vの間で変化するか確認する.
確認できたら電源を外し,周辺回路への電源電圧の変動の影響や磁気ノイズを減らすために電源とGNDの間にセラミックコンデンサー(0.1$\mu$)を接続する.
回路が完成したらそれぞれのICをソケットに取り付けテスターを接続し入力電源と出力電源を測定する.
\begin{figure}[htbp]
 \begin{center}
  \includegraphics[width=0.8\linewidth]{fig5.png}
 \end{center}
 \caption{NOTゲート回路}
 \label{fig:5}
\end{figure}

\subsection{Result}
図\ref{fig:1},図\ref{fig:2},図\ref{fig:3},図\ref{fig:4}は実験より得られた$V_{in}$,$V_{out}$の関係をグラフにしたものである.
全てのICにおいて入力電圧($V_{in}$)が5V付近では出力電圧($V_{out}$)は0Vを示し,入力電圧($V_{in}$)が0V付近では出力電圧($V_{out}$)は5Vを示を示した.
またICによって$V_{in}$を下げていった際の$V_{out}$の電圧の上がり方の鋭さが違っていた.74LS04,74HC04,74HC14の順で出力電圧の上がり方は鋭くなっていった.
また74HC14に関しては$V_{in}$を上げていく場合と下げていく場合で異なる動作をした.
$V_{in}$を下げていった場合では2Vを境として$V_{out}$が5V付近を示したが,$V_{in}$を上げていった場合は3Vを境として$V_{out}$が0Vとなった.
\begin{figure}[htbp]
 \begin{center}
  \includegraphics[width=0.8\linewidth]{fig1.png}
 \end{center}
 \caption{74LS04}
 \label{fig:1}
\end{figure}

\begin{figure}[htbp]
 \begin{center}
  \includegraphics[width=0.7\linewidth]{fig2.png}
 \end{center}
 \caption{74HC04}
 \label{fig:2}
\end{figure}


\begin{figure}[htbp]
 \begin{center}
  \includegraphics[width=0.7\linewidth]{fig3.png}
 \end{center}
 \caption{74HC14($V_{in}$下げていく)}
 \label{fig:3}
\end{figure}

\begin{figure}[htbp]
 \begin{center}
  \includegraphics[width=0.7\linewidth]{fig4.png}
 \end{center}
 \caption{74HC14($V_{in}$上げていく)}
 \label{fig:4}
\end{figure}

\subsection{Discussion}
今回はどのICについても入力電圧が5V付近の時出力電圧は0V付近を示し,逆に入力電圧が0付近の時出力電圧は5V付近を示したのでNOTゲートが正しく機能したと推測できる.

またTTLにおけるNOTゲートの動作を考察する.
\begin{figure}[htbp]
 \begin{center}
  \includegraphics[width=0.7\linewidth]{fig6.png}
 \end{center}
 \caption{TTL NOTgate}
 \label{fig:6}
\end{figure}

TTLにおけるNOTゲートは図\ref{fig:6}のように表せる.
この回路において入力電圧$V_{in}$が0Vの時トランジスタQ1にいおいてベースエミッタ電圧$V{BE}$はスイッチング電圧である0.6Vを超えるためQ1は作動する.
この時Q2において$V_{BE}$は0VとなるためQ2は作動せず抵抗による電圧降下が起こらないため出力電が5Vとなると考えられる.

一方,入力電圧$V_{in}$が5V付近の時Q1において$V_{BE}$は逆バイアスとなりベースエミッタ間には電流は流れなくなる.
この時ベースコレクタ間に電流が流れることとなりトランジスタQ2は作動し抵抗により電圧降下を起こすこととなる.
したがって$V_{out}$は0Vを示すと考えられる.

次にCMOSにおけるNOTゲートの動作を考察する.
\begin{figure}[htbp]
 \begin{center}
  \includegraphics[width=0.7\linewidth]{fig7.png}
 \end{center}
 \caption{TTL NOTgate}
 \label{fig:7}
\end{figure}

図\ref{fig:7}のCOMOS回路ではpチャネル型(Q1)とnチャネル型(Q2)の二つのMOSFETを組み合わせて作られている.
それぞれ入力電圧$V_{in}$が0V付近の時はQ1がオンとなり出力電圧$V_{out}$は5Vとなり,入力電圧$V_{in}$が5V付近の時Q2がオンとなり出力電圧$V_{out}$は0V付近になることが予想される.

以上の考察から74LS04を用いた回路が他の三つのICを用いた回路よりも入出力電圧のグラフが緩やかであったのはトランジスタにおいてスイッチング電圧を境目として完全に電流が流れる流れないの関係が成り立つのではなく,スイッチング電圧の付近で少しずつ電流が流れ始めるためであると予想できる.

またCMOS回路においては回路内部で入力電圧によって選択的にスイッチのようにon,offのように作動するのでトランジスタを用いたTTL回路に比べて鋭い入出力電圧のグラフが得られたと考えられる.

% ヒステリシスの考察





%=============================================================
\newpage
\end{document}

\documentclass[11pt, a4paper]{jsarticle}
\usepackage{multicol}  % パッケージの追加
\usepackage[dvipdfmx]{graphicx}
\begin{document}
%=============================================================
%=============================================================
\section{Diffraction from circular apertures and slits}
\subsection*{Purpose}
実験の目的は円形開口やスリットの干渉模様の観察し,干渉から開口の直径,スリット幅,ダブルスリットの間隔を計算することである.
\subsection{Circular aperture}
\subsubsection{Procedure}
図\ref{fig:five}に示すように光学系を組み立てる.
またそれぞれの距離を図\ref{fig:five}に示すように名前をつける.
まず,レーザーの前に円形開口を置きそこから離れた距離にスクリーンを設置する.
今回は$02mm$,$0.4mm$の二つの円形開口を用いて実験を行う.
また,スクリーンは方眼紙で製作する.
次に開口からクリーンまでの距離$L$を測定する.
レーザーの電源をつけるとスクリーンに円形の干渉縞を得る.
次に円形の干渉縞の中心から1番目の暗線のまでの距離$r$を測定する.
これを二種類の円形開口に対して距離$L$を変えながら二回ずつ計測を行う.
\begin{figure}[htbp]
 \begin{center}
  \includegraphics[width=100mm]{fig5.png}
 \end{center}
 \caption{円形開口の光学系}
 \label{fig:five}
\end{figure}\\

またこれらの測定した距離は式(\ref{eq:b})の関係をみたす.
\begin{equation}
    sin\theta = \frac{1.22\lambda}{D} \label{eq:b}
\end{equation}\\


またこの時$\theta$,$D$の値は非常に小さいので$sin\theta \simeq tan\theta = r/L$とみなすことができる.
また今回使用したHe-Ne Laserの波長は$\lambda = 632.8nm$である.
以上より計測値を関係式へ代入して円形開口の直径を推測する.
その後0.2mm円形開口を光学顕微鏡によって計測しその直径を実際に求めた.

\subsubsection{Result}
測定結果から次の表が得られた.
\begin{table}[htb]
 \begin{minipage}{0.45\hsize}
  \begin{center}
    \caption{$0.2mm$円形開口}
    \begin{tabular}{rrr} \hline
        $r(mm)$ & $L(mm)$ & $D(mm)$  \\ \hline
        0.25     & 583 & 0.18\\
        0.45    & 1435 & 0.246\\ \hline
    \end{tabular}
    \label{tab:b}
  \end{center}
 \end{minipage}
 \begin{minipage}{0.45\hsize}
  \begin{center}
    \caption{$0.4mm$円形開口}
    \begin{tabular}{rrr} \hline
        $r(mm)$ & $L(mm)$ & $D(mm)$  \\ \hline
        2.6   & 974 & 0.289\\
        3.5    & 1430 & 0.315\\ \hline
    \end{tabular}
    \label{tab:c}
  \end{center}
 \end{minipage}
\end{table} 


またそれぞれ以下のような干渉模様が観測できた.
\begin{figure}[htbp]
 \begin{minipage}{0.45\hsize}
  \begin{center}
   \includegraphics[width=60mm]{fig6.png}
  \end{center}
  \caption{$0.2mm$円形開口の干渉縞}
  \label{fig:six}
 \end{minipage}
 \begin{minipage}{0.45\hsize}
  \begin{center}
   \includegraphics[width=60mm]{fig7.png}
  \end{center}
  \caption{$0.4mm$円形開口の干渉縞}
  \label{fig:seven}
 \end{minipage}
\end{figure}

また0.2mm円形開口の実測値は0.19mmであった.
干渉模様は円の半径が大きくなっていくに連れて明暗の境目が不明瞭になって区別がつかなくなった.

\newpage
また以下は光学顕微鏡で観測した0.2mm円形開口の写真である.
図\ref{fig:27}における一目盛りは10${\mu}m$であるのでそれぞれの画像を透過させて円形開口の大きさを測定した.
その結果円形開口は0.19mmであることが測定された.

\begin{figure}[htbp]
 \begin{minipage}{0.45\hsize}
  \begin{center}
   \includegraphics[width=60mm]{fig26.png}
  \end{center}
  \caption{光学顕微鏡で観察した$0.2mm$円形開口}
  \label{fig:26}
 \end{minipage}
 \begin{minipage}{0.45\hsize}
  \begin{center}
   \includegraphics[width=60mm]{fig27.png}
  \end{center}
  \caption{光学顕微鏡で計測した目盛り}
  \label{fig:27}
 \end{minipage}
\end{figure}
\subsubsection{Discussion}
$0.2mm$円形開口の実験値と実測値である0.19mmとの差は比較的小さかった.一方で$0.4mm$円形開口の理論値との差はかなりの開きがあった.
また円形開口のプレートは複数個の開口があったためにそれぞれの円形が偏っていたりしたことも考えられる.
さらに他の要因としてそもそも$D$を算出する際に$sin\theta \simeq tan\theta = r/L$と近似したために実際の値とは異なる結果となった可能性が考えられる.
さらには目測による測定に誤差があったなどの要因も考えられる.
また干渉には開口部の大きさがスクリーンまでの距離に対して十分小さい時に起こるフレネル回折,ビーム源もしくは観測点がビームを回折するものから無限遠に位置する時に起こるフラウンホーファー回折などがあるが今回の実験では観測点はビームを回折する開口に比べて十分大きいと考えられるのでフラウンホーファー回折であると考えられる.
%=============================================================
\subsection{Single Slit}
\subsubsection{Procedure}
基本的には前回と同じ光学系を組み立てる.
今回は円形開口の代わりにシングルスリットをレーザーの前に置く.
以下の図はシングルスリットの光学系である.
今回は$r$,$L$の距離を測定する事によって$\omega$の値を推測する.
またこの実験では$0.1mm$,$0.2mm$二つのスリットを用いてそれぞれ1回ずつ測定を行う.
\begin{figure}[htbp]
 \begin{center}
  \includegraphics[width=100mm]{fig8.png}
 \end{center}
 \caption{シングルスリットの光学系}
 \label{fig:eight}
\end{figure}\\

またこれらの測定した距離は
\begin{equation}
    sin\theta = \frac{\lambda}{\omega} \label{eq:c}
\end{equation}\\
式(\ref{eq:c})の関係をみたす.
またこの時$\theta$,$D$の値は非常に小さいので$sin\theta \simeq tan\theta = r/L$とみなすことができる.
また今回使用したHe-Ne Laserの波長は$\lambda = 632.8nm$である.
以上より計測値を関係式へ代入してシングルスリットの間隔を推測する.
今回は時間の関係上実際に光学顕微鏡を用いてスリット幅を測定することはできなかった.

\subsubsection{Result}
測定結果から次の表が得られた.
また式(\ref{eq:c})の関係よりスリット幅が大きくなるにつれて干渉縞の間隔は小さくなっていくことが予想される.実際に実験からもこの関係が確かめられた.
\begin{table}[htb]
 \begin{minipage}{0.45\hsize}
  \begin{center}
    \caption{$0.1mm$シングルスリット}
    \begin{tabular}{rrr} \hline
        $r(mm)$ & $L(mm)$ & $\omega(mm)$  \\ \hline
        7.0    & 974 & 0.088\\ \hline
    \end{tabular}
    \label{tab:d}
  \end{center}
 \end{minipage}
 \begin{minipage}{0.45\hsize}
  \begin{center}
    \caption{$0.2mm$シングルスリット}
    \begin{tabular}{rrr} \hline
        $r(mm)$ & $L(mm)$ & $\omega(mm)$  \\ \hline
        3.5    & 974 & 0.176\\ \hline
    \end{tabular}
    \label{tab:e}
  \end{center}
 \end{minipage}
\end{table}

またそれぞれ以下のような干渉模様が観測できた.
\begin{figure}[htbp]
 \begin{minipage}{0.45\hsize}
  \begin{center}
   \includegraphics[width=60mm]{fig9.png}
  \end{center}
  \caption{$0.1mm$シングルスリットの干渉縞}
  \label{fig:nine}
 \end{minipage}
 \begin{minipage}{0.45\hsize}
  \begin{center}
   \includegraphics[width=60mm]{fig10.png}
  \end{center}
  \caption{$0.2mm$シングルスリットの干渉縞}
  \label{fig:ten}
 \end{minipage}
\end{figure}
\subsubsection{Discussion}
いずれの結果も理論値よりも$0.02mm$ほど小さくなってしまったのでどちらも目測による計測の時実際の感覚よりも大きく読んでしまった可能性が考えられる.
また光学顕微鏡を用いて実測値を計算していないので実際にスリット幅が0.2mm,0.1mmではなかったために誤差が生じていることも考えられる.
%=============================================================
\subsection{Yong Double Slit}
\subsubsection{Procedure}
光学系を図\ref{fig:eleven}に示すように製作する.
ダブルスリットをビームの経路に置きビーム光を回折させスクリーンに干渉縞を映し観察する.

\begin{figure}[htbp]
 \begin{center}
  \includegraphics[width=100mm]{fig11.png}
 \end{center}
 \caption{ダブルスリットの光学系}
 \label{fig:eleven}
\end{figure}

またこの時以下の式(\ref{eq:d}),(\ref{eq:f})の関係が成り立つ.それぞれ測定した$\Delta \theta$,$\Delta x$を式に代入することで二つのスリット間隔$d$
を推測する.
また今回は$0.2mm$,$0.1mm$の二つのスリット間隔をを持つダブルスリットに対して一回ずつ計測を行う.
また今回も時間の都合上ダブルスリットの間隔を光学顕微鏡を用いて計りはしなかった.

\begin{equation}
    \Delta\theta = \frac{\Delta x}{R} \label{eq:d}
\end{equation}
\begin{equation}
    \Delta\theta = \frac{\lambda}{d} \label{eq:f}
\end{equation}

\subsubsection{Result}
測定結果から次の表が得られた.

\begin{table}[htb]
 \begin{minipage}{0.45\hsize}
  \begin{center}
    \caption{$0.1mm$ダブルスリット}
    \begin{tabular}{rrr} \hline
        $\Delta x(mm)$ & $R(mm)$ & $d(mm)$  \\ \hline
        7.0    & 974 & 0.088\\ \hline
    \end{tabular}
    \label{tab:f}
  \end{center}
 \end{minipage}
 \begin{minipage}{0.45\hsize}
  \begin{center}
    \caption{$0.2mm$ダブルスリット}
    \begin{tabular}{rrr} \hline
        $\Delta x(mm)$ & $R(mm)$ & $d(mm)$  \\ \hline
        2.8    & 974 & 0.22\\ \hline
    \end{tabular}
    \label{tab:g}
  \end{center}
 \end{minipage}
\end{table}

また以下のような干渉縞が観察された.

\begin{figure}[htbp]
 \begin{minipage}{0.45\hsize}
  \begin{center}
   \includegraphics[width=60mm]{fig12.png}
  \end{center}
  \caption{$0.1m$ダブルスリットの干渉縞}
  \label{fig:12}
 \end{minipage}
 \begin{minipage}{0.45\hsize}
  \begin{center}
   \includegraphics[width=60mm]{fig13.png}
  \end{center}
  \caption{$0.2m$ダブルスリットの干渉縞}
  \label{fig:13}
 \end{minipage}
\end{figure}

\subsubsection{Discussion}
どちらの誤差も$-0.02mm$であったので目測の際に$r$を大きく読んでしまったことなどが考えられる.
またシングルスリットの時と同じような干渉縞が得られたがシングルスリットの方が中心が明るく,ダブルスリットの時は中心から離れた干渉縞でも強度が高かったことが観測された.これはダブルスリットの際は二つのスリットを通過した際にビーム光が素元波となり互いの波が干渉をするためだと考えられる.

%==========================================================
%==========================================================
\newpage
\end{document}

\documentclass[11pt, a4paper]{jsarticle}
\usepackage{multicol}  % パッケージの追加
\usepackage[dvipdfmx]{graphicx}
\begin{document}
%=============================================================
%=============================================================
\section{Michelson Interferometer and Coherence}
\subsection*{Purpose}
この実験の目的はマイケルソン干渉系によって作られる干渉模様を観測することと,実験で使用しているHe-Neレーザーのコヒーレント長をマイケルソン干渉系を用いて測定することである.
\subsection{Michelson Interferometer}
\subsubsection{Procedure}
まず以下のように干渉系を組み立てる.
\begin{figure}[htbp]
 \begin{center}
  \includegraphics[width=100mm]{fig14.png}
 \end{center}
 \caption{マイケルソン干渉系}
 \label{fig:14}
\end{figure}\\

まず光の強度を弱めるためにNDフィルター(25\%)をレーザーの前に設置する.
次にビームエキスパンダーを組み立てる.
今回はガリレオ型のエキスパンダーを組み立てる.
また,それぞれのミラーの高さを合わせて跳ね返った光がスクリーン上で一致するように調整を行う.
次にM2のミラーを移動させることで干渉縞が観測されたので干渉模様を写真に収める.
さらにその状態から机を叩く事により干渉模様がどのように変化するかを観察する.
その後エキスパンダーの二個目の凸レンズを移動させて二つのレンズ間距離を小さくする事で干渉模様にどのような変化が起きたかを観察する.
得られた干渉縞の写真をカメラに収める.
またこの時$BS-M1 = 8.0cm$,$BS-M2 = 5.0cm$の距離であった.

\subsubsection{Result}
二つの光がスクリーン上で重なり干渉を起こす事で図\ref{fig:15}のような干渉模様が得られた.
また机を叩いて揺れを起こす事でスクリーン上に映っていた干渉模様が消えた.
その後0.98[s]後に元どおりの干渉模様をまた形成した.
さらにエキスパンダーのレンズ間距離を小さくすると干渉模様が図\ref{fig:16}のように円の一部分のように丸くなって映った.
さらにレンズ間距離を元の位置よりも大きくすると同じような円状の干渉模様が得られた.

\begin{figure}[htbp]
 \begin{minipage}{0.45\hsize}
  \begin{center}
   \includegraphics[width=60mm]{fig15.png}
  \end{center}
  \caption{干渉模様}
  \label{fig:15}
 \end{minipage}
 \begin{minipage}{0.45\hsize}
  \begin{center}
   \includegraphics[width=60mm]{fig16.png}
  \end{center}
  \caption{レンズ間を狭くした時の干渉模様}
  \label{fig:16}
 \end{minipage}
\end{figure}

\subsubsection{Discussion}
まずM2の位置を調節する事でスクリーン上に平行な干渉模様が得られたからM1に反射された光とM2に反射された光がそれぞれ平面波であってその波がスクリーン上でぶつかる事によって干渉模様が得られたと考えられる.
この事実から実験で使用したレーザーはコヒーレントな光であると言える.

机を叩いた際に干渉模様が見えなくなったのは叩いた振動によって一時的にそれぞれの光路長が異なりスクリーン上で干渉条件が満たされなくなったためだと考えられる.

またミラーの角度を変えるとスクリーン上で観測される干渉模様の縞の感覚が異なったのは二つのビーム光の入社角度が異なるためだと考えられる.

さらにエキスパンダーのレンズ間距離を縮めた際に同心円状の干渉模様が得られたのは平行光だったビームエキスパンダーによって発散してビーム光が球面波になった為だと考えられる.

またマイケルソン干渉系は二つのビーム光の経路差に強く依存するため光路の途中にガラスなどを挿入することでガラスの屈折率を測定することに応用ができると予測される.

%=============================================================
\subsection{Coherence}
\subsubsection{Procedure}
前回の実験と同様に図\ref{fig:14}の光学系を製作する.初めの$BS-M1 = 8.0cm$,$BS-M2 = 5.0cm$の状態から$BS-M1$の距離のみを次第に広げていきスクリーン状で干渉模様が得られなくなるまで広げていきコヒーレント長を決定する.
\subsubsection{Result}
$BS-M1 = 10.0cm$の時は干渉模様が観測された.しかし$BS-M1 = 12.5cm$にすると干渉模様は見れなくなった.
以上の結果より可干渉光路差は$(|BS-M1|-|BS-M2|) \times 2 = 4.0cm$である事がわかった.
\subsubsection{Discussion}
一般に実験用のビーム光のコヒーレント長は数十cmであるので今回の実験から得られた値とは大きな開きがあった.
この原因としては二つの光の交差する角度が小さかったために干渉模様が得られなかった,調節不足で本当はあったのに観測できなかったなどの理由が考えられる.

%==========================================================
%==========================================================
\newpage
\end{document}

\documentclass[11pt, a4paper,twocolumn]{jarticle}
\usepackage[dvipdfmx]{graphicx}
\usepackage{docmute}
\begin{document}
%=============================================================
\section{Frequency filters \\based on operational amplifiers ($4^{th} \& 5^{th} day$)}
\subsubsection{Purpose}
増幅器を使いlow-pass filter, high-pass filter 作り,それを用いて周波数時間特性を調べる.
\subsection{Equipment}
\begin{itemize}
    \item 実験3と同様のもの
\end{itemize}
\subsubsection{Procedure}
まず図\ref{fig:12}に示すようにLow-pass-filterを作る.
この時オペアンプのパワーラインとGNDに前回同様0.1$\mu$Fのコンデンサを挟むことを確認する.
今回の実験ではファンクションジェネレーターを用いて矩形波,正弦波を作る.
まず$V_{in}$にファンクションジェネレーターを用いて矩形波を入力とし,10Hzから10KHzまで10倍づつ変化させながら測定した.
この時オシロスコープを用いて矩形波の入力電圧と出力電圧を同時に表示して観察した.
またこの時のデータをUSBに保存した.
次に正弦波を同様の手順で測定し記録した.

次に図\ref{fig:13}に示すようにHigh-pass-filterを作り,前回同様ファンクションジェネレーターにより矩形波の入力電圧と出力電圧の関係を測定し,次に正弦波の入力電圧と出力電圧の関係を測定した.

最後に図\ref{fig:14}に示すようにBand-pass-filterを作り,前回同様に矩形波,正弦波の入力電圧においてそれぞれ周波数を変えながら測定を行なった.

これらの実験から得られた結果を対数グラフにプロットすることによって周波数と振幅の関係について調べる.
また出力電圧の波形と入力電圧の波形を見比べてその位相差について調べる.

\begin{figure}[htbp]
 \begin{center}
  \includegraphics[width=0.8\linewidth]{fig12.png}
 \end{center}
 \caption{High pass filter}
 \label{fig:12}
\end{figure}

\begin{figure}[htbp]
 \begin{center}
  \includegraphics[width=0.8\linewidth]{fig13.png}
 \end{center}
 \caption{Low pass filter}
 \label{fig:13}
\end{figure}

\begin{figure}[htbp]
 \begin{center}
  \includegraphics[width=0.8\linewidth]{fig14.png}
 \end{center}
 \caption{Band pass filter}
 \label{fig:14}
\end{figure}

\subsubsection{Result}
測定の結果まず,Low pass filterについて以下のような結果が得られた.
まずは矩形波についての結果をしめす.

\begin{figure}[htbp]
 \begin{center}
  \includegraphics[width=0.8\linewidth]{fig31.png}
 \end{center}
 \caption{Low pass filter(Vin 10Hz)}
 \label{fig:31}
\end{figure}

\begin{figure}[htbp]
 \begin{center}
  \includegraphics[width=0.8\linewidth]{fig32.png}
 \end{center}
 \caption{Low pass filter(Vin 100Hz)}
 \label{fig:32}
\end{figure}

\begin{figure}[htbp]
 \begin{center}
  \includegraphics[width=0.8\linewidth]{fig33.png}
 \end{center}
 \caption{Low pass filter(Vin 1kHz)}
 \label{fig:33}
\end{figure}

\begin{figure}[htbp]
 \begin{center}
  \includegraphics[width=0.8\linewidth]{fig34.png}
 \end{center}
 \caption{Low pass filter(Vin 10kHz)}
 \label{fig:34}
\end{figure}

\newpage

続いて正弦波についての結果を示す.

\begin{figure}[htbp]
 \begin{center}
  \includegraphics[width=0.8\linewidth]{fig35.png}
 \end{center}
 \caption{Low pass filter(Vin 10Hz)}
 \label{fig:35}
\end{figure}

\begin{figure}[htbp]
 \begin{center}
  \includegraphics[width=0.8\linewidth]{fig36.png}
 \end{center}
 \caption{Low pass filter(Vin 100Hz)}
 \label{fig:36}
\end{figure}

\begin{figure}[htbp]
 \begin{center}
  \includegraphics[width=0.8\linewidth]{fig37.png}
 \end{center}
 \caption{Low pass filter(Vin 1kHz)}
 \label{fig:37}
\end{figure}

\begin{figure}[htbp]
 \begin{center}
  \includegraphics[width=0.8\linewidth]{fig38.png}
 \end{center}
 \caption{Low pass filter(Vin 10kHz)}
 \label{fig:38}
\end{figure}

\newpage

さらにHigh pass filterについても以下のような結果が得られた.
まずは矩形波の結果を示す.

\begin{figure}[htbp]
 \begin{center}
  \includegraphics[width=0.8\linewidth]{fig39.png}
 \end{center}
 \caption{High pass filter(Vin 10Hz)}
 \label{fig:39}
\end{figure}

\begin{figure}[htbp]
 \begin{center}
  \includegraphics[width=0.8\linewidth]{fig40.png}
 \end{center}
 \caption{High pass filter(Vin 100Hz)}
 \label{fig:40}
\end{figure}

\begin{figure}[htbp]
 \begin{center}
  \includegraphics[width=0.8\linewidth]{fig41.png}
 \end{center}
 \caption{High pass filter(Vin 1kHz)}
 \label{fig:41}
\end{figure}

\begin{figure}[htbp]
 \begin{center}
  \includegraphics[width=0.8\linewidth]{fig42.png}
 \end{center}
 \caption{High pass filter(Vin 10kHz)}
 \label{fig:42}
\end{figure}

\newpage

続いて正弦波の結果を以下に示す.

\begin{figure}[htbp]
 \begin{center}
  \includegraphics[width=0.8\linewidth]{fig43.png}
 \end{center}
 \caption{High pass filter(Vin 10Hz)}
 \label{fig:43}
\end{figure}

\begin{figure}[htbp]
 \begin{center}
  \includegraphics[width=0.8\linewidth]{fig44.png}
 \end{center}
 \caption{High pass filter(Vin 100Hz)}
 \label{fig:44}
\end{figure}

\begin{figure}[htbp]
 \begin{center}
  \includegraphics[width=0.8\linewidth]{fig45.png}
 \end{center}
 \caption{High pass filter(Vin 1kHz)}
 \label{fig:45}
\end{figure}

\begin{figure}[htbp]
 \begin{center}
  \includegraphics[width=0.8\linewidth]{fig46.png}
 \end{center}
 \caption{High pass filter(Vin 10kHz)}
 \label{fig:46}
\end{figure}

\newpage

最後にBand pass filterについても以下のような結果が得られた.
まずは矩形波の結果を示す.

\begin{figure}[htbp]
 \begin{center}
  \includegraphics[width=0.8\linewidth]{fig47.png}
 \end{center}
 \caption{Band pass filter(Vin 10Hz)}
 \label{fig:47}
\end{figure}

\begin{figure}[htbp]
 \begin{center}
  \includegraphics[width=0.8\linewidth]{fig48.png}
 \end{center}
 \caption{Band pass filter(Vin 100Hz)}
 \label{fig:48}
\end{figure}

\begin{figure}[htbp]
 \begin{center}
  \includegraphics[width=0.8\linewidth]{fig49.png}
 \end{center}
 \caption{Band pass filter(Vin 2kHz)}
 \label{fig:49}
\end{figure}

\begin{figure}[htbp]
 \begin{center}
  \includegraphics[width=0.8\linewidth]{fig50.png}
 \end{center}
 \caption{Band pass filter(Vin 10kHz)}
 \label{fig:50}
\end{figure}

\newpage

続いて正弦波の結果を以下に示す.

\begin{figure}[htbp]
 \begin{center}
  \includegraphics[width=0.8\linewidth]{fig51.png}
 \end{center}
 \caption{Band pass filter(Vin 10Hz)}
 \label{fig:51}
\end{figure}

\begin{figure}[htbp]
 \begin{center}
  \includegraphics[width=0.8\linewidth]{fig52.png}
 \end{center}
 \caption{Band pass filter(Vin 100Hz)}
 \label{fig:52}
\end{figure}

\begin{figure}[htbp]
 \begin{center}
  \includegraphics[width=0.8\linewidth]{fig53.png}
 \end{center}
 \caption{Band pass filter(Vin 1kHz)}
 \label{fig:53}
\end{figure}

\begin{figure}[htbp]
 \begin{center}
  \includegraphics[width=0.8\linewidth]{fig54.png}
 \end{center}
 \caption{Band pass filter(Vin 5kHz)}
 \label{fig:54}
\end{figure}

\begin{figure}[htbp]
 \begin{center}
  \includegraphics[width=0.8\linewidth]{fig55.png}
 \end{center}
 \caption{Band pass filter(Vin 10kHz)}
 \label{fig:55}
\end{figure}

\newpage


\documentclass[11pt, a4paper,twocolumn]{jarticle}
\usepackage[dvipdfmx]{graphicx}
\begin{document}

実験結果よりLow pass filterでは入力電圧の周波数が小さい時は元の矩形波を反転させた形の出力が得られたが,周波数が上がるにつれて元の矩形波が変化した際に遅れを生じるようになり指数関数的に追いつくような波形が得られた.
特に1kHzでの出力波は三角波のようになり,それ以上周波数が上がると出力は一定値となった.

また,High pass filterでは入力電圧の周波数が高くなるにつれて元の矩形を反転させたような出力が得られた.また低周波数領域では入力電圧が変化した瞬間だけ一瞬電圧の大きさがが大きくなりその後指数関数的に減少するような結果となった.

さらにBand pass filterでは10Hz付近では正弦波の重ね合わせのような出力が得られた.
これらの波の振幅は非常に小さかった.
また,100Hz付近になると入力の矩形波を反転させたような出力が得られた.
さらに周波数をあげると出力は一定値となった.

またそれぞれの場合において正弦波の入力を与えるとLow pass filterでは高周波数領域で振幅が小さくなり,High pass filterでは低周波数領域で振幅が小さくなり,Band pass filterでは低周波数領域と高周波数領域で振幅が小さくなった.
また総じてカットされる周波数での出力波は滑らかではなく不連続な波となった.

\subsubsection{Discussion}
今回はなぜ低周波数領域や高周波数領域の入力をカットできるのかを考察していく.
まず,図\ref{fig:13},図\ref{fig:14},図\ref{fig:15}における回路は前回の実験で行なった反転増幅回路とほぼ同じ構造をしているのがわかる.
そのことより今回の出力は入力を反転させたような波形であると予測できる.

Low pass filterについて考察を行う.
図\ref{fig:13}の回路においてキルヒホッフの法則を用いると以下の微分方程式を立てることができる.
\begin{equation}
    RC\frac{dV_{out}}{dt} + V_{out} = -V_{in}
\end{equation}

この微分方程式をとくと以下のような式を得る.
\begin{equation}
    V_{out} = -u(t)\left[1-exp\left(-\frac{t}{RC}\right)\right]
\end{equation}

この式よりtが十分大きい時expの部分は0に近づくため出力は入力の反転した値が出てくる.
一方で高周波の場合はexpは1に近づくので出力は0に近づいていくこととなり実験結果と一致することが確かめられた.

次にHigh pass filterについて考察を行う.
図\ref{fig:14}の回路においてキルヒホッフの法則を用いると以下の微分方程式を立てることができる.
\begin{equation}
    V_{out} = -RC\frac{d}{dt}(V_{out}+V_{in})
\end{equation}

この微分方程式をとくと以下のような式を得る.
\begin{equation}
    V_{out} = -u(t)exp\left(-\frac{t}{RC}\right)\
\end{equation}

この式よりtが十分小さい時expの値は1に近づくこととなり出力は入力の反転した値となる.
またtが十分大きい時はexpの値は0に近づくこととなり結果的に出力は0となる.
この性質は高周波数成分を通し,低周波数成分をカットする性質であるので実験の値とも一致することが確かめられた.

ここで得られた特定の周波数をカットするフィルターの応用例について考えてみる.
例えば,画像や音声において人間の目に見えないもしくは聞き取ることができない周波数成分はデータとして余計なのでフィルターに通してカットすることで,元のデータ量を落とすことができる.

\end{document}


%=============================================================
\newpage
\end{document}

\documentclass[11pt, a4paper,twocolumn]{jarticle}
\usepackage[dvipdfmx]{graphicx}
\usepackage{listings,jlisting}

\begin{document}
%=============================================================
\section{Measurement of the temperature dependence of the resistivity of $Bi_2Sr_2Ca_2Cu_3O_{10}$ as a superconductor($6^{th} day$)}

\subsection{Purpose}
超電導の転移温度の測定を通して超電導の原理について学ぶ.
\subsection{Procedure}
前回同様に図\ref{fig:29}のように温度依存性の測定のための実験装置を組み立てる.
次に超電導リボン($Bi_2Sr_2Ca_2Cu_3O_{10}$)の電圧-電流特性を四端子測定法で測定する.
この時超電導のヒステリシス特性を観察するために室温から超電導が起こる相転移温度まで下げていった場合と,超電導状態から室温に近づけていった場合の二パターンについて測定を行いそれぞれの結果を温度-抵抗値のグラフにプロットしていく.
測定の際は電流が100mAを超えないように注意して行う.

\subsection{Result}
測定の結果温度と抵抗値の関係をプロットすると以下のようなグラフが得られた.
ヒステリシスが測定され,室温から温度を下げていった場合よりも超電導状態から室温にあげていった時は結果が全体的に左にシフトした.
また実験結果より温度下降時は相転移温度は140Kほど,上昇時は130Kほどであることが確認された.

\begin{figure}[htbp]
 \begin{center}
  \includegraphics[width=0.8\linewidth]{fig39.png}
 \end{center}
 \caption{温度下降}
 \label{fig:39}
\end{figure}

\subsection{Discussion}
超電導が起こる理由について考える.
まず超電導が起きる極低温状態においては原子の熱振動は十分小さくなっていることが予想される.そこで一つの電子が原子に衝突すると原子が動かされ,プラスに帯電している原子は別の電子を引き寄せるという原理によって結果として電子同士が引き寄せられる現象が図\ref{fig:40}が起きる.
以上のようにして電子のペアについて片方が原子に衝突しても,その衝撃を片方がプラス,もう一方がマイナスに分散吸収してペア全体としては何事もなかったように進むことによって抵抗を感じずに電流を流すことができる.

次に超電導の応用について考える.
超電導状態では小さい電圧で多くの電流を流すことができるので非常に強い磁場を作り出すことが可能である.これはリニアモーターカーなどで利用される.
また量子コンピュータにおいては量子状態がノイズによって壊れることが問題であるためノイズを小さくするために超電導状態が用いられる.
もし超電導状態を室温で実現することができればリニアモータカーが新幹線よりもコストの低い乗り物になることができ,量子コンピュータの製作コストが格段に低くなることが予想される.

\begin{figure}[htbp]
 \begin{center}
  \includegraphics[width=0.8\linewidth]{fig40.png}
 \end{center}
 \caption{超電導メカニズム}
 \label{fig:40}
\end{figure}
%=============================================================
\newpage
\end{document}

%=============================================================
\end{document}
