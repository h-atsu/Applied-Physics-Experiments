\documentclass[11pt, a4paper,twocolumn]{jarticle}
\usepackage[dvipdfmx]{graphicx}
\begin{document}

実験結果よりLow pass filterでは入力電圧の周波数が小さい時は元の矩形波を反転させた形の出力が得られたが,周波数が上がるにつれて元の矩形波が変化した際に遅れを生じるようになり指数関数的に追いつくような波形が得られた.
特に1kHzでの出力波は三角波のようになり,それ以上周波数が上がると出力は一定値となった.

また,High pass filterでは入力電圧の周波数が高くなるにつれて元の矩形を反転させたような出力が得られた.また低周波数領域では入力電圧が変化した瞬間だけ一瞬電圧の大きさがが大きくなりその後指数関数的に減少するような結果となった.

さらにBand pass filterでは10Hz付近では正弦波の重ね合わせのような出力が得られた.
これらの波の振幅は非常に小さかった.
また,100Hz付近になると入力の矩形波を反転させたような出力が得られた.
さらに周波数をあげると出力は一定値となった.

またそれぞれの場合において正弦波の入力を与えるとLow pass filterでは高周波数領域で振幅が小さくなり,High pass filterでは低周波数領域で振幅が小さくなり,Band pass filterでは低周波数領域と高周波数領域で振幅が小さくなった.
また総じてカットされる周波数での出力波は滑らかではなく不連続な波となった.

\subsubsection{Discussion}
今回はなぜ低周波数領域や高周波数領域の入力をカットできるのかを考察していく.
まず,図\ref{fig:13},図\ref{fig:14},図\ref{fig:15}における回路は前回の実験で行なった反転増幅回路とほぼ同じ構造をしているのがわかる.
そのことより今回の出力は入力を反転させたような波形であると予測できる.

Low pass filterについて考察を行う.
図\ref{fig:13}の回路においてキルヒホッフの法則を用いると以下の微分方程式を立てることができる.
\begin{equation}
    RC\frac{dV_{out}}{dt} + V_{out} = -V_{in}
\end{equation}

この微分方程式をとくと以下のような式を得る.
\begin{equation}
    V_{out} = -u(t)\left[1-exp\left(-\frac{t}{RC}\right)\right]
\end{equation}

この式よりtが十分大きい時expの部分は0に近づくため出力は入力の反転した値が出てくる.
一方で高周波の場合はexpは1に近づくので出力は0に近づいていくこととなり実験結果と一致することが確かめられた.

次にHigh pass filterについて考察を行う.
図\ref{fig:14}の回路においてキルヒホッフの法則を用いると以下の微分方程式を立てることができる.
\begin{equation}
    V_{out} = -RC\frac{d}{dt}(V_{out}+V_{in})
\end{equation}

この微分方程式をとくと以下のような式を得る.
\begin{equation}
    V_{out} = -u(t)exp\left(-\frac{t}{RC}\right)\
\end{equation}

この式よりtが十分小さい時expの値は1に近づくこととなり出力は入力の反転した値となる.
またtが十分大きい時はexpの値は0に近づくこととなり結果的に出力は0となる.
この性質は高周波数成分を通し,低周波数成分をカットする性質であるので実験の値とも一致することが確かめられた.

ここで得られた特定の周波数をカットするフィルターの応用例について考えてみる.
例えば,画像や音声において人間の目に見えないもしくは聞き取ることができない周波数成分はデータとして余計なのでフィルターに通してカットすることで,元のデータ量を落とすことができる.

\end{document}
