\documentclass[11pt, a4paper,twocolumn]{jarticle}
\usepackage[dvipdfmx]{graphicx}
\usepackage{listings,jlisting}

\begin{document}
%=============================================================
\section{定常状態での光信号の取り込み}
\subsection{目的}
この実験では定常状態(走査しない状態)での入力物体と取得信号の関係について検証する.

\subsection{手順}
図\ref{fig:5}のような光学濃度の異なる物体の中心に対して得られる信号を測定した.
次に物体のエッジに対して得られる信号を測定した.
ここで左上のサンプルから光学濃度の大きい順に1,2,...,11とラベリングすることにした.
また今回の測定はデータ数100点として測定を行いそのデータの平均を求めた.
\begin{figure}[ht]
 \begin{center}
  \includegraphics[width=0.8\linewidth]{fig5.png}
 \end{center}
 \caption{サンプル}
 \label{fig:5}
\end{figure}

\subsection{結果}
測定の結果中心に対しての光強度は表\ref{fig:hoge}のようになった.
またエッジ周辺の光信号の取り込みにおいてはエッジに集光スポット(測定点)が乗った時にサンプル番号1と同じ電圧が得られ,集光スポットがエッジから外れるとそのサンプルの中心と同じ電圧が得られた.

\begin{table}[ht]
\centering
\caption{光信号の取り込み}
\label{my-label}
\begin{tabular}{c c}
\hline
サンプル番号 & 測定電圧[V] \\ \hline
1 & 0.594286267 \\
2 & 0.684387733 \\
3 & 0.895845505 \\
4 & 1.018323762 \\
5 & 1.128849337 \\
6 & 1.37754902 \\
7 & 1.736388356 \\
8 & 1.819731168 \\
9 & 1.856152525 \\
10 & 1.924024248 \\
11 & 1.954663822\\
\end{tabular}
\label{fig:hoge}
\end{table}

\newpage


\subsection{考察}
まず表\ref{fig:hoge}の結果より黒いサンプルほど測定された電圧は低くなり白いサンプルになるほど測定電圧は大きな値となった.
これは黒い物体ほど光を吸収しやすやすく,白い物体ほど光を反射しやすいという事実と相違ない.
次にエッジ付近での信号強度について考察する.
エッジに集光スポットが乗った時にサンプル番号1の測定電圧と同じ強度の信号が得られたことよりエッジの集光スポットはエッジの太さ0.5mmよりも小さいことが考えられる.


%=============================================================
\newpage
\end{document}
