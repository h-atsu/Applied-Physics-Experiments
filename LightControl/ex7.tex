\documentclass[11pt, a4paper,twocolumn]{jarticle}
\usepackage[dvipdfmx]{graphicx}
\usepackage{listings,jlisting}

\begin{document}
%=============================================================
\section{走査プログラムの作成と光パターン観察}
\subsection{目的}
今回の実験では前回作成した光学系で光信号を取得するための制御プログラムを作成する.
\subsection{手順}
まず,プログラムを実行すると同時にレーザーが発光し走査が開始され,データをテキストファイルとしてパソコンに書き込んだ後,消光するプログラムを作成した.
なお,実行時にコマンドプロンプトで読み取るデータ点数,サンプリング周波数を設定できるようにした.
次に,ミラーの回転により,入力物体に生成される光スポットの形状がどのように変化するかを観察した.
\subsection{結果}
光スポットの形状は,コリメートされた光が焦点距離100mmの凸レンズの中心付近を通った際には十分集光されたが,凸レンズの中心を外れる程と楕円形に広がって集光されなくなった.

\subsection{考察}
今回実験に用いたレンズは球面平凸レンズだったので,レンズの中心を通る光は十分焦点に集まるがレンズの中心を離れるほど光がぼけていく原因は球面収差によるものであると考えられる.
%=============================================================
\newpage
\end{document}
