\documentclass[11pt, a4paper,twocolumn]{jarticle}
\usepackage[dvipdfmx]{graphicx}
\usepackage{listings,jlisting}

\begin{document}
%=============================================================
\section{光信号の取り込み系の作制}
\subsection{目的}
この実験では最終的な実験光学系の可動ミラーを除いた光学系を作成することにより一点の光信号を取り込むことを可能にすることが目的である.

\subsection{手順}
図\ref{fig:a}のように光学系を組み立てた.
まず,半導体レーザーを焦点距離40mmの凸レンズを用いて平行光にしてからハーフミラーを通過させ,焦点距離100mmの凸レンズによって集光させた.
今度は入力物体から発せられる点光源が逆の光路を辿ることによりコリメートされた光がハーフミラーで反射され,焦点距離40mmの凸レンズを用いることでフォトダイオードの感光素子に集光されるようにした.
またフォトディテクターから得られた電圧をパソコンにテキストファイルとして書き込むプログラムを作成した.
このプログラムも前セメスターで行ったデジタル計測のプログラムを流用して使った.

\begin{figure}[ht]
 \begin{center}
  \includegraphics[width=0.8\linewidth]{fig4.png}
 \end{center}
 \caption{光取り込み系の動作確認のための光学系}
 \label{fig:a}
\end{figure}

\subsection{結果}
初めの週の実験では黒い紙と白い紙を交互に入力してもフォトダイオードからの電圧差を観測することができなかった.
二週間目の実験では紙の色に応じてフォトダイオードの出力電圧が変わることが確認できた.

\subsection{考察}
まず,初めの実験で入力物体に応じた電圧差を観測できなかった理由はビームエキスパンダーを作った際にうまくコリメートできていなかったために集光スポットが大きくなってしまったからだと考えられる.
次に半導体レーザーの方位角の設定について考える.
半導体レーザーは方位角によって発散角が異なっており楕円形のビーム広がりを持つ.
後の実験においては縦縞の周期的な光信号を読み取ることが目的なので回折限界の式(一次元走査光学系の特性評価)より平行光の直径が大きいほど集光スポットは小さくなる.
従って今後の実験においては楕円の長軸を水平にした状態でレーザーを設置することが望ましいと考えられる.


%=============================================================
\newpage
\end{document}
