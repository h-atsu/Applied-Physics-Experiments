\documentclass[11pt, a4paper,twocolumn]{jarticle}
\usepackage[dvipdfmx]{graphicx}
\usepackage{amssymb}
\begin{document}
%=============================================================
\section{溶媒粘性と楕円球にかかる放射圧の測定 (5日目)}
\subsection{実験目的}
今回の実験目的はポリスチレン球のトラッピングによりエタノールの粘性を計算し,その値を理科年表と比較することと楕円型のポリスチレン球を用いたトラッピングを行い球形とどのような違いがあるのかを観察することである.
\subsection{実験手順}
ポリスチレン球をエタノールに分散させ,前回と同じくトラッピングできる最大速度を求めたのち4日目の結果を参考にして式\ref{eq:3}よりエタノールの粘性を求めた.
この時トラップ力Fを図\ref{fig:4}の直線近似から求め代入した.

次に楕円型のポリスチレン球を純水に分散させて同様にトラッピングできる最大速度を求める.この際に楕円粒子のアスペクト比によるトラッピング性能の違いを観察する.

\begin{equation}
    \eta = \frac{F}{3\pi av}
\label{eq:3}
\end{equation}

\subsection{結果}
それぞれの大きさにおける測定結果とエタノール粘性推定値は表\ref{table:5},表\ref{table:6},表\ref{table:7}のようになった.
室温におけるエタノールの粘性理想値は$1.08 \times{10^{-3}}$であるから全体的に理想値よりも大きい値を出した.

また楕円粒子のトラッピングは出力電流が0.8Aから1.0Aの時捕捉でき,それ以外の電流の時はトラップ力が強すぎたり弱すぎたりしたためにトラップすることができなかった.
またモニターで確認した際のアスペクト比が1.36:1の楕円粒子はトラッピングに成功した一方で2:1の楕円粒子のトラッピングはできなかった.
またどの楕円粒子もトラッピングした直後にモニター上で円形に変形した.

\begin{table}[htbp]
    \begin{center}
        \scalebox{0.8}[0.9]{ %ココ
        \begin{tabular}{cccc}
            光強度[mW] & FPPS & トラップ力[N] & エタノール粘性 \\ \hline
            19.56  & 13000 & $4.111\times 10^{-12}$  & $1.678\times 10^{-3}$  \\
            49.72 & 32000 & $1.0147\times 10^{-11}$  & $1.681\times 10^{-3}$ \\
            81.45 & 50000 & $1.649\times 10^{-11}$  & $1.750\times 10^{-3}$
        \end{tabular}
        }
        \caption{5$\mu m$の粘性測定}
        \label{table:5}
    \end{center}
\end{table}

\begin{table}[htbp]
    \begin{center}
        \scalebox{0.8}[0.9]{ %ココ
        \begin{tabular}{cccc}
            光強度[mW] & FPPS & トラップ力[N] & エタノール粘性 \\ \hline
            19.741624  & 8000 & $5.948\times 10^{-12}$  & $1.972\times 10^{-3}$  \\
            49.847876 & 23000 & $1.197\times 10^{-11}$  & $1.380\times 10^{-3}$ \\
            86.42324 & 36000 & $1.928\times 10^{-11}$  & $1.421\times 10^{-3}$
        \end{tabular}
        }
        \caption{10$\mu m$の粘性測定}
        \label{table:6}
    \end{center}
\end{table}

\begin{table}[htbp]
    \begin{center}
        \scalebox{0.8}[0.9]{ %ココ
        \begin{tabular}{cccc}
            光強度[mW] & FPPS & トラップ力[N] & エタノール粘性 \\ \hline
            20.48806  & 27000 & $2.749\times 10^{-12}$  & $1.350\times 10^{-3}$  \\
            50.034485 & 74000 & $5.703\times 10^{-12}$  & $1.022\times 10^{-3}$ \\
            92.332525 & 101000 & $9.933\times 10^{-12}$  & $1.304\times 10^{-3}$
        \end{tabular}
        }
        \caption{2$\mu m$の粘性測定}
        \label{table:7}
    \end{center}
\end{table}

\subsection{考察}
エタノールの粘度が全体的に大きな値を出したことについてはサンプルポリスチレン球の粒径がそれぞれ異なっていたことやトラッピング成功の判定基準が曖昧であったために実際の最大速度よりも小さな値を採用してしまったことなどが考えられる.
より正確な粘度を推定するためには粒径の大きさの精度を高くする,トラッピング判定を複数回行う,検量線を作る際のサンプル数を増やすなどが考えられる.
また今回の実験では溶媒がエタノールだったので揮発による水面の揺れによるノイズが加わった可能性が考えられる.
そのためガラスボトムディッシュよりも密閉性の高い容器に入れて測定する方法などが考えられる.

次に楕円粒子のトラッピングの際にモニター上で楕円粒子が円になったのは楕円内をレーザーが通過した際に対称性が最も高くなるのが楕円粒子が直立した時であるためだと考えられる.
そのため横向きの楕円粒子についてもレーザー光が入射した瞬間に直立するような方向のモーメントが働きトラップ状態では常に直立状態あると考えられる.
またアスペクト比が大きくなるにつれてトラッピングが難しくなった理由はアスペクト比の増加に伴い球に比べて力が等方的に加わらなくなり不安定になるために台の揺れやトラッピングの移動によってトラッピング状態がすぐ解除されるためだと考えられる.

%=============================================================
\newpage
\end{document}
