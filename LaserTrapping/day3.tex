\documentclass[11pt, a4paper,twocolumn]{jarticle}
\usepackage[dvipdfmx]{graphicx}
\usepackage{amssymb}
\begin{document}
%=============================================================
\section{光放射圧の測定 (3日目)}
\subsection{実験目的}
今回の実験目的はレーザーによるビーズの捕捉力をストークスの式を利用して求めることである.

\subsection{実験手順}
まず,前回同様にレーザー光によって直径10$\mu m$ポリスチレン球が捕捉できることを確認した.このときビーズを左右に動かしても捕捉が外れずに引きずっていないことに注意した.
次にステッピングモーターの終端測度を早くしていきトラッピングの限界速度を測定した.
最後に測定結果より式\ref{eq:1}に示す,ストークスの式を用いてレーザートラッピングの力を算出した.
ここで流体の粘度を$\eta$,粒子直径をa,粒子速度をvとした.
また今回は溶媒に純水を使用したので$\eta=1.002$として計算を行った.

\begin{equation}
\label{eq:1}
    F = 3\pi{\eta}av
\end{equation}


\subsection{結果}
測定の結果,最大速度は表\ref{table:1}のようになった.
このときレーザー駆動電流は1.452Aであった.
また測定結果からトラッピング力は$8.688\times 10^{-12}[N]$となった.

\begin{table}[htbp]
    \begin{center}
        \begin{tabular}{cccc}
            LPPS & FPPS & Rmsec & 結果 \\ \hline
            100  & 2300 & 2000  & \checkmark  \\
            100  & 2400 & 2000  & $\times$  \\
            9999 & 2300 & 2000  & \checkmark
        \end{tabular}
        \caption{トラッピング限界速度の測定}
        \label{table:1}
    \end{center}
\end{table}


\subsection{考察}
まず表\ref{table:1}のLPPSが9999のときにもトラッピングが成功していることより静止状態から動かした時の慣性力はポリスチレン球のトラッピングに影響を与えておらず流体の粘性による抵抗力のみが寄与していることが考えられる.

また実験においてはじめにステッピングモータで上下左右に移動させた際に他の沈んだポリスチレン球に当たってもトラップが外れずに通過させることができたので,トラップ状態においては水中でつり合った状態で浮遊しておりガラス底に押し当てて引きずった状態ではないと考えられる.

ステッピングモータの移動速度を上げて行った際にトラッピングできなくなるのは空気抵抗のように,溶媒が粘度を持っており速度に比例した進行方向逆向きの力が働くためだと考えられる.

高速にステッピングモータを駆動させてもトラッピング状態を続けるためには式\ref{eq:1}よりレーザー光強度を上げる,粒径を小さくする,粘度の小さい溶媒を用いるなどが考えられる.

さらに式\ref{eq:1}を用いるためにはレイノズル定数$R=\rho av/\eta$が十分小さい必要があるが今回の実験ではR=7.8$\times 10^{-8}$であるので十分小さいと言え,今後の実験においてもストークスの式を用いることは妥当であると言える.
ここで$\rho$は溶媒の密度を表す.



%=============================================================
\newpage
\end{document}
