\documentclass[11pt, a4paper,twocolumn]{jarticle}
\usepackage[dvipdfmx]{graphicx}
\usepackage{amssymb}
\begin{document}
%=============================================================
\section{マクロファージのトラッピング (6日目)}
\subsection{実験目的}
今回の実験目的はマクロファージのトラッピングを通して微生物を任意の場所に移動させることである.

\subsection{実験手順}
前回までの実験と同様にマクロファージの入った溶液をガラスボトムディッシュに設置しマクロファージのトラッピングが行えるかを確認した.
次にマクロファージをトラップした状態での最大速度を求めその時の様子を観察した.
またその時の対物レンズ上でのレーザー光強度を測定した.


\subsection{結果}
トラッピングの最大速度はFPPSが4000でこの時レンズ付近のレーザー光強度は4.51mWであった.またレーザー光強度が7.03mWの時も測定を行ったがこの時はマクロファージがガラス底にくっつきトラップすることができなかった.またトラップする際は上下左右でトラップ力がことなっった.
さらに,マクロファージをトラップして移動させた際に移動する方向に応じてマクロファージの向きが変わった.

\subsection{考察}
トラップ力がポリスチレン球に比べて著しく弱くなったのはマクロファージ内部の屈折率が一様では無くレーザー光による力が等方的に働かなかったためだと考えられる.
さらにトラップして動かした際にマクロファージの向きが変わったのは屈折率の変化に加えてマクロファージの細胞自体が伸び縮みしたためだと考えられる.
またマクロファージは細胞であるのでポリスチレン球と比べガラス底にくっつきやすためにポリスチレン球に比べてトラップが困難であっと考えられる.
実際にマクロファージなどの細胞をトラップした状態でラマン顕微鏡などで分析する際には,レーザー光強度が大きすぎると観測前に細胞が死んでしまうことが考えられるのでより小さいレーザー光強度でトラッピングする必要があると考えられる.


%=============================================================
\newpage
\end{document}
