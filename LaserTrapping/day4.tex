\documentclass[11pt, a4paper,twocolumn]{jarticle}
\usepackage[dvipdfmx]{graphicx}
\usepackage{amssymb}
\begin{document}
%=============================================================
\section{光放射圧のレーザー光強度依存性,粒子との関係 (4日目)}
\subsection{実験目的}
今回の実験目的はポリスチレン球の粒径,レーザー光強度の二つのパラメータとトラップ力の関係性について確かめる.

\subsection{実験手順}
まず直径4.78${\mu}m$のポリスチレン球を水に分散させレーザー光強度を3段階に変化させながら前回同様にトラップできる最大速度を測定した.
次に同様の実験を直径10${\mu}m$,2${mu}m$に対して行い,レーザー強度を横軸,トラップ力を縦軸にグラフを制作する.


\subsection{結果}
測定の結果4.78${\mu}m$,10${\mu}m$,2${\mu}m$,最大速度,レーザー光強度はそれぞれ表\ref{table:2},表\ref{table:3}表\ref{table:4}のようになった.
またそれぞれのレーザー強度,トラップ力の関係は図\ref{fig:4}のようになった.
各測定におけるレーザー光強度は実験2で求めた直線近似の一次式に代入することにより求めたが電流出力強度が0.69Aを下回ると光強度が負になるため,その際は測定データと近い値を使った.

\begin{table}[htbp]
    \begin{center}
        \begin{tabular}{ccc}
            レーザー強度 & FPPS & トラップ力  \\ \hline
            0.623 & 1500 & $2.833^{-13}$   \\
            21.73 & 2200 & $4.155^{-12}$   \\
            51.65 & 6000 & $1.133^{-11}$  \\
            81.45 & 8500 & $1.605^{-11}$
        \end{tabular}
        \caption{4.78$\mu m$の測定結果}
        \label{table:2}
    \end{center}
\end{table}

\begin{table}[htbp]
    \begin{center}
        \begin{tabular}{ccc}
            レーザー強度 & FPPS & トラップ力  \\ \hline
            19.56 & 12000 & $4.533^{-12}$   \\
            50.97 & 32000 & $1.209{-11}$   \\
            84.87 & 43000 & $1.162{-11}$  \\
        \end{tabular}
        \caption{10$\mu m$の測定結果}
        \label{table:3}
    \end{center}
\end{table}

\begin{table}[htbp]
    \begin{center}
        \begin{tabular}{ccc}
            レーザー強度[mW] & FPPS & トラップ力[N]  \\ \hline
            19.24 & 30000 & $2.266{-12}$   \\
            50.35 & 80000 & $6.044^{-12}$   \\
            82.32 & 110000 & $8.310^{-12}$  \\
        \end{tabular}
        \caption{2$\mu m$の測定結果}
        \label{table:4}
    \end{center}
\end{table}

\begin{figure}[htbp]
 \begin{center}
  \includegraphics[width=0.8\linewidth]{fig4.png}
 \end{center}
 \caption{レーザー強度とトラップ力の関係}
 \label{fig:4}
\end{figure}

\subsection{考察}
実験結果よりレーザー光強度とトラップ力の関係は比例の関係があると言える.
さらに2日目の実験結果よりレーザー光強度と出力電圧の間にも比例の関係があったと考えられるのでトラップ力は出力電圧に比例すると考えられる.

また結果より粒径が10$\mu m$,5$\mu m$の際はレーザー光強度とトラッピング力の関係が同じ傾きを示したの対し,2$\mu m$の時に傾きが小さくなったのはレーザーの集光スポットよりもポリスチレン球の大きさが小さくなりレーザー光の力を十分に受けることができなかった可能性が考えられる.
1日目の考察よりレーザー光の分解能は$d=\lambda/{2NA}$で与えられ,今回使用した水侵レンズのの開口数NA=1.15,レーザー波長$\lambda=532nm$を代入するとd=$1.89 \times 10^{-7}nm$となりポリスチレン球よりも小さいが,2日目の考察よりガラスボトムディッシュ上のポリスチレン球でのレーザー光で考えると10倍の大きさになりポリスチレン球よりも大きくなっている可能性は十分考えられる.




%=============================================================
\newpage
\end{document}
