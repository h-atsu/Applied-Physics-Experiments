\documentclass[11pt, a4paper,twocolumn]{jarticle}
\usepackage[dvipdfmx]{graphicx}
\usepackage{listings,jlisting}
\usepackage{docmute}
\begin{document}
\begin{titlepage}
  \begin{center}
    {\Huge 基礎デジタル計測}\\
    \vspace{10truept}
    {\Huge Basics of digital measurements}\\
    \vspace{30truept}
    {\huge   提出者 : 08A17153 羽田充宏}\\ % 学籍番号
    {\huge 共同実験者 : 08A17154 畠中俊平}\\ % 学籍番号
    \vspace{50truept}

    \begin{list}{}{\setlength{\leftmargin}{95pt}}
    \item {\huge 実験実施日 : 2019年6月14日}\\
    \vspace{10truept}
    \item {\huge 実験実施日 : 2019年6月17日}\\
    \vspace{10truept}
    \item {\huge 実験実施日 : 2019年6月21日}\\
    \vspace{10truept}
    \item {\huge 実験実施日 : 2019年6月24日}\\
    \vspace{10truept}
    \item {\huge 実験実施日 : 2019年6月28日}\\
    \vspace{10truept}
    \item {\huge 実験実施日 : 2019年7月1日}\\
    \vspace{40truept}

    \end{list}
    \vspace{50truept}

  \end{center}
\end{titlepage}


% template======================================================
% \section{}
% \subsection{}
% \subsubsection{Purpose}
% \subsection{Equipment}
% \begin{itemize}
%     \item
% \end{itemize}
% \subsubsection{Procedure}
% \subsubsection{Result}
% \subsubsection{Discussion}
%=============================================================
% \begin{figure}[ht]
%  \begin{center}
%   \includegraphics[width=0.8\linewidth]{fig1.png}
%  \end{center}
%  \caption{74HC14}
%  \label{fig:1}
% \end{figure}
% \clearpage
%=============================================================
\twocolumn
[
\begin{center}
    \textbf{\Large Introduction}\\
    \begin{flushleft}
        この実験ではコンピュータやその周辺機器を使った実験の制御方法を学ぶ.
        また実験を通してC言語を用いてコンピュータや電気回路を制御する方法を学習する.
        こうしたスキルを使うことで,話し手によりマイクロフォンより発せられる音声のキャプチャーをを開発し,制御工学を学ぶ.
        さらに,キャプチャーした音声を信号処理することで周波数フィルターを作製する.
    \end{flushleft}
    \begin{itemize}
        \item パソコンを使ってのC言語学習 (1st day).
        \item AD変換,DA変換やC言語を用いたAD/DA変換を学習する (2nd day).
        \item 音声キャプチャーのための電気回路を製作する (3rd day).
        \item GNU Octaveの学習および理解をした後,キャプチャーした音声の高速フーリエ変換(FFT)を観察する (4th day).
        \item 周波数領域における信号処理 (5-6th day).
    \end{itemize}
\end{center}
\vspace{30truept}
]
%=============================================================
\documentclass[11pt, a4paper,twocolumn]{jarticle}
\usepackage[dvipdfmx]{graphicx}
\begin{document}
%=============================================================
\section{Characteristics of \\NOT gates ($1^{st} day$)}
\subsection{Purpose}
異なるICによる論理ゲートの挙動を実験する.
また与えられた電気回路を計画的に製作し,NOTゲートの入出力電圧を計測する.
\subsection{Equipment}
\begin{itemize}
    \item NOT gate IC(74LS04,74HC04,74HC14)
    \item Universal printed circuit board ICB-86
    \item Switching power supply with a connector
    \item Connector for power suuply
    \item Trinmming potentiometer 1k$\Omega$
    \item 14pin IC socket
    \item 8 pin IC socket
    \item Ceramic capacitor 0.1$\mu$
    \item Dogital multimeters
    \item Clip test leads
\end{itemize}
\subsection{Procedure}
今回は半田付けによって論理回路を実装していく.
NOTゲート回路を図\ref{fig:5}のように作る.
まず,スイッチング電源からの5V,GNDは8ピンソケットを使って基盤中央の縦のラインに給電する.
可変抵抗を基盤に取り付け,ジャンパー線と半田によって回路を作っていく.
今回は一つのNOTゲートのみ使えれば良いので1,2番ピンのゲートを使えるように14ピンICソケットを配線する.
また使わないゲートの入力端子はGNDまたは5Vに接続し,出力端子は解放する.
次に電源と接続し可変抵により入力電圧が0~5Vの間で変化するか確認する.
確認できたら電源を外し,周辺回路への電源電圧の変動の影響や磁気ノイズを減らすために電源とGNDの間にセラミックコンデンサー(0.1$\mu$)を接続する.
回路が完成したらそれぞれのICをソケットに取り付けテスターを接続し入力電源と出力電源を測定する.
\begin{figure}[htbp]
 \begin{center}
  \includegraphics[width=0.8\linewidth]{fig5.png}
 \end{center}
 \caption{NOTゲート回路}
 \label{fig:5}
\end{figure}

\subsection{Result}
図\ref{fig:1},図\ref{fig:2},図\ref{fig:3},図\ref{fig:4}は実験より得られた$V_{in}$,$V_{out}$の関係をグラフにしたものである.
全てのICにおいて入力電圧($V_{in}$)が5V付近では出力電圧($V_{out}$)は0Vを示し,入力電圧($V_{in}$)が0V付近では出力電圧($V_{out}$)は5Vを示を示した.
またICによって$V_{in}$を下げていった際の$V_{out}$の電圧の上がり方の鋭さが違っていた.74LS04,74HC04,74HC14の順で出力電圧の上がり方は鋭くなっていった.
また74HC14に関しては$V_{in}$を上げていく場合と下げていく場合で異なる動作をした.
$V_{in}$を下げていった場合では2Vを境として$V_{out}$が5V付近を示したが,$V_{in}$を上げていった場合は3Vを境として$V_{out}$が0Vとなった.
\begin{figure}[htbp]
 \begin{center}
  \includegraphics[width=0.8\linewidth]{fig1.png}
 \end{center}
 \caption{74LS04}
 \label{fig:1}
\end{figure}

\begin{figure}[htbp]
 \begin{center}
  \includegraphics[width=0.7\linewidth]{fig2.png}
 \end{center}
 \caption{74HC04}
 \label{fig:2}
\end{figure}


\begin{figure}[htbp]
 \begin{center}
  \includegraphics[width=0.7\linewidth]{fig3.png}
 \end{center}
 \caption{74HC14($V_{in}$下げていく)}
 \label{fig:3}
\end{figure}

\begin{figure}[htbp]
 \begin{center}
  \includegraphics[width=0.7\linewidth]{fig4.png}
 \end{center}
 \caption{74HC14($V_{in}$上げていく)}
 \label{fig:4}
\end{figure}

\subsection{Discussion}
今回はどのICについても入力電圧が5V付近の時出力電圧は0V付近を示し,逆に入力電圧が0付近の時出力電圧は5V付近を示したのでNOTゲートが正しく機能したと推測できる.

またTTLにおけるNOTゲートの動作を考察する.
\begin{figure}[htbp]
 \begin{center}
  \includegraphics[width=0.7\linewidth]{fig6.png}
 \end{center}
 \caption{TTL NOTgate}
 \label{fig:6}
\end{figure}

TTLにおけるNOTゲートは図\ref{fig:6}のように表せる.
この回路において入力電圧$V_{in}$が0Vの時トランジスタQ1にいおいてベースエミッタ電圧$V{BE}$はスイッチング電圧である0.6Vを超えるためQ1は作動する.
この時Q2において$V_{BE}$は0VとなるためQ2は作動せず抵抗による電圧降下が起こらないため出力電が5Vとなると考えられる.

一方,入力電圧$V_{in}$が5V付近の時Q1において$V_{BE}$は逆バイアスとなりベースエミッタ間には電流は流れなくなる.
この時ベースコレクタ間に電流が流れることとなりトランジスタQ2は作動し抵抗により電圧降下を起こすこととなる.
したがって$V_{out}$は0Vを示すと考えられる.

次にCMOSにおけるNOTゲートの動作を考察する.
\begin{figure}[htbp]
 \begin{center}
  \includegraphics[width=0.7\linewidth]{fig7.png}
 \end{center}
 \caption{TTL NOTgate}
 \label{fig:7}
\end{figure}

図\ref{fig:7}のCOMOS回路ではpチャネル型(Q1)とnチャネル型(Q2)の二つのMOSFETを組み合わせて作られている.
それぞれ入力電圧$V_{in}$が0V付近の時はQ1がオンとなり出力電圧$V_{out}$は5Vとなり,入力電圧$V_{in}$が5V付近の時Q2がオンとなり出力電圧$V_{out}$は0V付近になることが予想される.

以上の考察から74LS04を用いた回路が他の三つのICを用いた回路よりも入出力電圧のグラフが緩やかであったのはトランジスタにおいてスイッチング電圧を境目として完全に電流が流れる流れないの関係が成り立つのではなく,スイッチング電圧の付近で少しずつ電流が流れ始めるためであると予想できる.

またCMOS回路においては回路内部で入力電圧によって選択的にスイッチのようにon,offのように作動するのでトランジスタを用いたTTL回路に比べて鋭い入出力電圧のグラフが得られたと考えられる.

% ヒステリシスの考察





%=============================================================
\newpage
\end{document}

\documentclass[11pt, a4paper]{jsarticle}
\usepackage{multicol}  % パッケージの追加
\usepackage[dvipdfmx]{graphicx}
\begin{document}
%=============================================================
%=============================================================
\section{Diffraction from circular apertures and slits}
\subsection*{Purpose}
実験の目的は円形開口やスリットの干渉模様の観察し,干渉から開口の直径,スリット幅,ダブルスリットの間隔を計算することである.
\subsection{Circular aperture}
\subsubsection{Procedure}
図\ref{fig:five}に示すように光学系を組み立てる.
またそれぞれの距離を図\ref{fig:five}に示すように名前をつける.
まず,レーザーの前に円形開口を置きそこから離れた距離にスクリーンを設置する.
今回は$02mm$,$0.4mm$の二つの円形開口を用いて実験を行う.
また,スクリーンは方眼紙で製作する.
次に開口からクリーンまでの距離$L$を測定する.
レーザーの電源をつけるとスクリーンに円形の干渉縞を得る.
次に円形の干渉縞の中心から1番目の暗線のまでの距離$r$を測定する.
これを二種類の円形開口に対して距離$L$を変えながら二回ずつ計測を行う.
\begin{figure}[htbp]
 \begin{center}
  \includegraphics[width=100mm]{fig5.png}
 \end{center}
 \caption{円形開口の光学系}
 \label{fig:five}
\end{figure}\\

またこれらの測定した距離は式(\ref{eq:b})の関係をみたす.
\begin{equation}
    sin\theta = \frac{1.22\lambda}{D} \label{eq:b}
\end{equation}\\


またこの時$\theta$,$D$の値は非常に小さいので$sin\theta \simeq tan\theta = r/L$とみなすことができる.
また今回使用したHe-Ne Laserの波長は$\lambda = 632.8nm$である.
以上より計測値を関係式へ代入して円形開口の直径を推測する.
その後0.2mm円形開口を光学顕微鏡によって計測しその直径を実際に求めた.

\subsubsection{Result}
測定結果から次の表が得られた.
\begin{table}[htb]
 \begin{minipage}{0.45\hsize}
  \begin{center}
    \caption{$0.2mm$円形開口}
    \begin{tabular}{rrr} \hline
        $r(mm)$ & $L(mm)$ & $D(mm)$  \\ \hline
        0.25     & 583 & 0.18\\
        0.45    & 1435 & 0.246\\ \hline
    \end{tabular}
    \label{tab:b}
  \end{center}
 \end{minipage}
 \begin{minipage}{0.45\hsize}
  \begin{center}
    \caption{$0.4mm$円形開口}
    \begin{tabular}{rrr} \hline
        $r(mm)$ & $L(mm)$ & $D(mm)$  \\ \hline
        2.6   & 974 & 0.289\\
        3.5    & 1430 & 0.315\\ \hline
    \end{tabular}
    \label{tab:c}
  \end{center}
 \end{minipage}
\end{table} 


またそれぞれ以下のような干渉模様が観測できた.
\begin{figure}[htbp]
 \begin{minipage}{0.45\hsize}
  \begin{center}
   \includegraphics[width=60mm]{fig6.png}
  \end{center}
  \caption{$0.2mm$円形開口の干渉縞}
  \label{fig:six}
 \end{minipage}
 \begin{minipage}{0.45\hsize}
  \begin{center}
   \includegraphics[width=60mm]{fig7.png}
  \end{center}
  \caption{$0.4mm$円形開口の干渉縞}
  \label{fig:seven}
 \end{minipage}
\end{figure}

また0.2mm円形開口の実測値は0.19mmであった.
干渉模様は円の半径が大きくなっていくに連れて明暗の境目が不明瞭になって区別がつかなくなった.

\newpage
また以下は光学顕微鏡で観測した0.2mm円形開口の写真である.
図\ref{fig:27}における一目盛りは10${\mu}m$であるのでそれぞれの画像を透過させて円形開口の大きさを測定した.
その結果円形開口は0.19mmであることが測定された.

\begin{figure}[htbp]
 \begin{minipage}{0.45\hsize}
  \begin{center}
   \includegraphics[width=60mm]{fig26.png}
  \end{center}
  \caption{光学顕微鏡で観察した$0.2mm$円形開口}
  \label{fig:26}
 \end{minipage}
 \begin{minipage}{0.45\hsize}
  \begin{center}
   \includegraphics[width=60mm]{fig27.png}
  \end{center}
  \caption{光学顕微鏡で計測した目盛り}
  \label{fig:27}
 \end{minipage}
\end{figure}
\subsubsection{Discussion}
$0.2mm$円形開口の実験値と実測値である0.19mmとの差は比較的小さかった.一方で$0.4mm$円形開口の理論値との差はかなりの開きがあった.
また円形開口のプレートは複数個の開口があったためにそれぞれの円形が偏っていたりしたことも考えられる.
さらに他の要因としてそもそも$D$を算出する際に$sin\theta \simeq tan\theta = r/L$と近似したために実際の値とは異なる結果となった可能性が考えられる.
さらには目測による測定に誤差があったなどの要因も考えられる.
また干渉には開口部の大きさがスクリーンまでの距離に対して十分小さい時に起こるフレネル回折,ビーム源もしくは観測点がビームを回折するものから無限遠に位置する時に起こるフラウンホーファー回折などがあるが今回の実験では観測点はビームを回折する開口に比べて十分大きいと考えられるのでフラウンホーファー回折であると考えられる.
%=============================================================
\subsection{Single Slit}
\subsubsection{Procedure}
基本的には前回と同じ光学系を組み立てる.
今回は円形開口の代わりにシングルスリットをレーザーの前に置く.
以下の図はシングルスリットの光学系である.
今回は$r$,$L$の距離を測定する事によって$\omega$の値を推測する.
またこの実験では$0.1mm$,$0.2mm$二つのスリットを用いてそれぞれ1回ずつ測定を行う.
\begin{figure}[htbp]
 \begin{center}
  \includegraphics[width=100mm]{fig8.png}
 \end{center}
 \caption{シングルスリットの光学系}
 \label{fig:eight}
\end{figure}\\

またこれらの測定した距離は
\begin{equation}
    sin\theta = \frac{\lambda}{\omega} \label{eq:c}
\end{equation}\\
式(\ref{eq:c})の関係をみたす.
またこの時$\theta$,$D$の値は非常に小さいので$sin\theta \simeq tan\theta = r/L$とみなすことができる.
また今回使用したHe-Ne Laserの波長は$\lambda = 632.8nm$である.
以上より計測値を関係式へ代入してシングルスリットの間隔を推測する.
今回は時間の関係上実際に光学顕微鏡を用いてスリット幅を測定することはできなかった.

\subsubsection{Result}
測定結果から次の表が得られた.
また式(\ref{eq:c})の関係よりスリット幅が大きくなるにつれて干渉縞の間隔は小さくなっていくことが予想される.実際に実験からもこの関係が確かめられた.
\begin{table}[htb]
 \begin{minipage}{0.45\hsize}
  \begin{center}
    \caption{$0.1mm$シングルスリット}
    \begin{tabular}{rrr} \hline
        $r(mm)$ & $L(mm)$ & $\omega(mm)$  \\ \hline
        7.0    & 974 & 0.088\\ \hline
    \end{tabular}
    \label{tab:d}
  \end{center}
 \end{minipage}
 \begin{minipage}{0.45\hsize}
  \begin{center}
    \caption{$0.2mm$シングルスリット}
    \begin{tabular}{rrr} \hline
        $r(mm)$ & $L(mm)$ & $\omega(mm)$  \\ \hline
        3.5    & 974 & 0.176\\ \hline
    \end{tabular}
    \label{tab:e}
  \end{center}
 \end{minipage}
\end{table}

またそれぞれ以下のような干渉模様が観測できた.
\begin{figure}[htbp]
 \begin{minipage}{0.45\hsize}
  \begin{center}
   \includegraphics[width=60mm]{fig9.png}
  \end{center}
  \caption{$0.1mm$シングルスリットの干渉縞}
  \label{fig:nine}
 \end{minipage}
 \begin{minipage}{0.45\hsize}
  \begin{center}
   \includegraphics[width=60mm]{fig10.png}
  \end{center}
  \caption{$0.2mm$シングルスリットの干渉縞}
  \label{fig:ten}
 \end{minipage}
\end{figure}
\subsubsection{Discussion}
いずれの結果も理論値よりも$0.02mm$ほど小さくなってしまったのでどちらも目測による計測の時実際の感覚よりも大きく読んでしまった可能性が考えられる.
また光学顕微鏡を用いて実測値を計算していないので実際にスリット幅が0.2mm,0.1mmではなかったために誤差が生じていることも考えられる.
%=============================================================
\subsection{Yong Double Slit}
\subsubsection{Procedure}
光学系を図\ref{fig:eleven}に示すように製作する.
ダブルスリットをビームの経路に置きビーム光を回折させスクリーンに干渉縞を映し観察する.

\begin{figure}[htbp]
 \begin{center}
  \includegraphics[width=100mm]{fig11.png}
 \end{center}
 \caption{ダブルスリットの光学系}
 \label{fig:eleven}
\end{figure}

またこの時以下の式(\ref{eq:d}),(\ref{eq:f})の関係が成り立つ.それぞれ測定した$\Delta \theta$,$\Delta x$を式に代入することで二つのスリット間隔$d$
を推測する.
また今回は$0.2mm$,$0.1mm$の二つのスリット間隔をを持つダブルスリットに対して一回ずつ計測を行う.
また今回も時間の都合上ダブルスリットの間隔を光学顕微鏡を用いて計りはしなかった.

\begin{equation}
    \Delta\theta = \frac{\Delta x}{R} \label{eq:d}
\end{equation}
\begin{equation}
    \Delta\theta = \frac{\lambda}{d} \label{eq:f}
\end{equation}

\subsubsection{Result}
測定結果から次の表が得られた.

\begin{table}[htb]
 \begin{minipage}{0.45\hsize}
  \begin{center}
    \caption{$0.1mm$ダブルスリット}
    \begin{tabular}{rrr} \hline
        $\Delta x(mm)$ & $R(mm)$ & $d(mm)$  \\ \hline
        7.0    & 974 & 0.088\\ \hline
    \end{tabular}
    \label{tab:f}
  \end{center}
 \end{minipage}
 \begin{minipage}{0.45\hsize}
  \begin{center}
    \caption{$0.2mm$ダブルスリット}
    \begin{tabular}{rrr} \hline
        $\Delta x(mm)$ & $R(mm)$ & $d(mm)$  \\ \hline
        2.8    & 974 & 0.22\\ \hline
    \end{tabular}
    \label{tab:g}
  \end{center}
 \end{minipage}
\end{table}

また以下のような干渉縞が観察された.

\begin{figure}[htbp]
 \begin{minipage}{0.45\hsize}
  \begin{center}
   \includegraphics[width=60mm]{fig12.png}
  \end{center}
  \caption{$0.1m$ダブルスリットの干渉縞}
  \label{fig:12}
 \end{minipage}
 \begin{minipage}{0.45\hsize}
  \begin{center}
   \includegraphics[width=60mm]{fig13.png}
  \end{center}
  \caption{$0.2m$ダブルスリットの干渉縞}
  \label{fig:13}
 \end{minipage}
\end{figure}

\subsubsection{Discussion}
どちらの誤差も$-0.02mm$であったので目測の際に$r$を大きく読んでしまったことなどが考えられる.
またシングルスリットの時と同じような干渉縞が得られたがシングルスリットの方が中心が明るく,ダブルスリットの時は中心から離れた干渉縞でも強度が高かったことが観測された.これはダブルスリットの際は二つのスリットを通過した際にビーム光が素元波となり互いの波が干渉をするためだと考えられる.

%==========================================================
%==========================================================
\newpage
\end{document}

\documentclass[11pt, a4paper]{jsarticle}
\usepackage{multicol}  % パッケージの追加
\usepackage[dvipdfmx]{graphicx}
\begin{document}
%=============================================================
%=============================================================
\section{Michelson Interferometer and Coherence}
\subsection*{Purpose}
この実験の目的はマイケルソン干渉系によって作られる干渉模様を観測することと,実験で使用しているHe-Neレーザーのコヒーレント長をマイケルソン干渉系を用いて測定することである.
\subsection{Michelson Interferometer}
\subsubsection{Procedure}
まず以下のように干渉系を組み立てる.
\begin{figure}[htbp]
 \begin{center}
  \includegraphics[width=100mm]{fig14.png}
 \end{center}
 \caption{マイケルソン干渉系}
 \label{fig:14}
\end{figure}\\

まず光の強度を弱めるためにNDフィルター(25\%)をレーザーの前に設置する.
次にビームエキスパンダーを組み立てる.
今回はガリレオ型のエキスパンダーを組み立てる.
また,それぞれのミラーの高さを合わせて跳ね返った光がスクリーン上で一致するように調整を行う.
次にM2のミラーを移動させることで干渉縞が観測されたので干渉模様を写真に収める.
さらにその状態から机を叩く事により干渉模様がどのように変化するかを観察する.
その後エキスパンダーの二個目の凸レンズを移動させて二つのレンズ間距離を小さくする事で干渉模様にどのような変化が起きたかを観察する.
得られた干渉縞の写真をカメラに収める.
またこの時$BS-M1 = 8.0cm$,$BS-M2 = 5.0cm$の距離であった.

\subsubsection{Result}
二つの光がスクリーン上で重なり干渉を起こす事で図\ref{fig:15}のような干渉模様が得られた.
また机を叩いて揺れを起こす事でスクリーン上に映っていた干渉模様が消えた.
その後0.98[s]後に元どおりの干渉模様をまた形成した.
さらにエキスパンダーのレンズ間距離を小さくすると干渉模様が図\ref{fig:16}のように円の一部分のように丸くなって映った.
さらにレンズ間距離を元の位置よりも大きくすると同じような円状の干渉模様が得られた.

\begin{figure}[htbp]
 \begin{minipage}{0.45\hsize}
  \begin{center}
   \includegraphics[width=60mm]{fig15.png}
  \end{center}
  \caption{干渉模様}
  \label{fig:15}
 \end{minipage}
 \begin{minipage}{0.45\hsize}
  \begin{center}
   \includegraphics[width=60mm]{fig16.png}
  \end{center}
  \caption{レンズ間を狭くした時の干渉模様}
  \label{fig:16}
 \end{minipage}
\end{figure}

\subsubsection{Discussion}
まずM2の位置を調節する事でスクリーン上に平行な干渉模様が得られたからM1に反射された光とM2に反射された光がそれぞれ平面波であってその波がスクリーン上でぶつかる事によって干渉模様が得られたと考えられる.
この事実から実験で使用したレーザーはコヒーレントな光であると言える.

机を叩いた際に干渉模様が見えなくなったのは叩いた振動によって一時的にそれぞれの光路長が異なりスクリーン上で干渉条件が満たされなくなったためだと考えられる.

またミラーの角度を変えるとスクリーン上で観測される干渉模様の縞の感覚が異なったのは二つのビーム光の入社角度が異なるためだと考えられる.

さらにエキスパンダーのレンズ間距離を縮めた際に同心円状の干渉模様が得られたのは平行光だったビームエキスパンダーによって発散してビーム光が球面波になった為だと考えられる.

またマイケルソン干渉系は二つのビーム光の経路差に強く依存するため光路の途中にガラスなどを挿入することでガラスの屈折率を測定することに応用ができると予測される.

%=============================================================
\subsection{Coherence}
\subsubsection{Procedure}
前回の実験と同様に図\ref{fig:14}の光学系を製作する.初めの$BS-M1 = 8.0cm$,$BS-M2 = 5.0cm$の状態から$BS-M1$の距離のみを次第に広げていきスクリーン状で干渉模様が得られなくなるまで広げていきコヒーレント長を決定する.
\subsubsection{Result}
$BS-M1 = 10.0cm$の時は干渉模様が観測された.しかし$BS-M1 = 12.5cm$にすると干渉模様は見れなくなった.
以上の結果より可干渉光路差は$(|BS-M1|-|BS-M2|) \times 2 = 4.0cm$である事がわかった.
\subsubsection{Discussion}
一般に実験用のビーム光のコヒーレント長は数十cmであるので今回の実験から得られた値とは大きな開きがあった.
この原因としては二つの光の交差する角度が小さかったために干渉模様が得られなかった,調節不足で本当はあったのに観測できなかったなどの理由が考えられる.

%==========================================================
%==========================================================
\newpage
\end{document}

\documentclass[11pt, a4paper,twocolumn]{jarticle}
\usepackage[dvipdfmx]{graphicx}
\usepackage{listings,jlisting}

\begin{document}
%=============================================================
\section{Temprature dependence of the resistance($4^{th} day$)}

\subsection{Purpose}
金属抵抗の温度による変化を学ぶ.
\subsection{Procedure}
まず試料を四端子測定法の測定回路に接続する.次に熱電対と共に試料を固定し液体窒素の入った容器の中に沈めていく.
この時試料の温度は熱電対の起電力によって測定する.
今回使用した熱電対はクロメルーアルメル熱電対であり測定部分を氷水の容器に浸し基準を273Kにして測定する.
測定器の製作は図\ref{fig:29}のように組み立てる.

以上の手順でCu,NiCr,Wの低効率を測定していく.
測定は室温から77K程度まで5段階に分けて測定を行なった.
測定結果より抵抗値を最小二乗法で求めその温度依存性をグラフに示す.

\begin{figure}[htbp]
 \begin{center}
  \includegraphics[width=0.8\linewidth]{fig29.png}
 \end{center}
 \caption{温度依存性の実験装置}
 \label{fig:29}
\end{figure}

\subsection{Result}
測定の結果温度と抵抗値の関係をプロットすると以下のようなグラフが得られた.

\begin{figure}[htbp]
 \begin{center}
  \includegraphics[width=0.8\linewidth]{fig30.png}
 \end{center}
 \caption{Wの温度依存}
 \label{fig:30}
\end{figure}

\begin{figure}[htbp]
 \begin{center}
  \includegraphics[width=0.8\linewidth]{fig31.png}
 \end{center}
 \caption{NiCrの温度依存}
 \label{fig:31}
\end{figure}

\begin{figure}[htbp]
 \begin{center}
  \includegraphics[width=0.8\linewidth]{fig32.png}
 \end{center}
 \caption{Cuの温度依存}
 \label{fig:32}
\end{figure}

\newpage


\subsection{Discussion}
実験結果より温度が低くなるにつれて抵抗率が小さくなった原因について考える.
まず前回の考察より金属中において電流の流れを阻害するものは熱振動する格子だと考えられることがわかった.したがって温度が下がるとその分熱運動により振動する格子の振動の激しさは穏やかになることが予想され結果として電流が格子と衝突する回数がすくなり流れる電流量が多くなり抵抗率が小さくなることが予想される.

%=============================================================
\newpage
\end{document}

\documentclass[11pt, a4paper,twocolumn]{jarticle}
\usepackage[dvipdfmx]{graphicx}

\begin{document}
%=============================================================
\section{Signal processing in frequency space (Highpass and lowpass filter) ($5-6^{th} day$)}
% ===============================================================
\subsection{Purpose}
今回の実験では取り込んだ音声信号の周波数空間(フーリエ空間)での信号処理を学ぶ.特定の周波数のみを通す簡単な周波数フィルターの実戦から,窓関数を用いたローパスフィルターおよびハイパスフィルターを学ぶ.
具体的には,異なる共鳴周波数を有する複数の音叉からの音声信号の中から,信号処理によって特定の周波数のみを取り出す.また人が発する高い声と低い声を同時に録音し,信号所によって高い声のみ,または低い声のみを取り出す.
% =======================================================
\subsection{Procedure}
\noindent
\textbf{Task 5.1 特定周波数の取り込み} \\
異なる周波数を有する音叉を二つ選びそれらを同時に鳴らし,音声信号を取得する.
その音声信号を一方のみを通す周波数フィルターをOctaveで政策しフーリエ変換した周波数領域の信号に掛け算する.
次にをれを逆フーリエ変換を行なってスピーカーで鳴らすことで正しく分離できたかを確認するとともにオシロスコープで波の振動数を計測する.
今回は330Hzと440Hzの音叉を同時に鳴らしてフィルターにより440Hzの音叉成分のみを取り出すようなフィルターを用意した.
また今回のサンプリング周波数は20kHz,サンプリング数は20000点とした.

\noindent
\textbf{Task 5.2 ハイパスフィルター,ローパスフィルター} \\
同時に高い声と低い声で異なる言葉を発生し,その声をコンピュータに取り込む.
取り込んだ音声信号をフーリエ変換する.
その後Octaveで窓関数を作り[0,F]までのデータ点数を0,[F+1,40000]までを1にするステップ関数をかけて逆フーリエ変換して低い声がカットされるような整数Fを探した.
今回はサンプリング周波数20kHz,サンプリング数40000点として測定した.
\noindent
\textbf{Task 5.3 ボイスチェンジャー} \\
声を取り込み,周波数を変化させることで,声の高さを変えてスピーカーで再生する.
% =======================================================
\subsection{Result}
\noindent
\textbf{Task 5.1 特定周波数の取り込み} \\
取り込んだ時間領域の声の信号は図\ref{fig:mix}のようになった.
また図\ref{fig:28}はそのうち最初の1000点のみを表示した.
またこの信号のフーリエ変換は図\ref{fig:fftmix}のようになった.
さらにこのフィルター後の関数を逆フーリエ変換したのちスピーカーで再生した.
さらにオシロスコープで周波数を確認すると440Hzの値を得た.

\begin{figure}[htbp]
 \begin{center}
  \includegraphics[width=0.8\linewidth]{mix.png}
 \end{center}
 \caption{330Hz,440Hz音叉}
 \label{fig:mix}
\end{figure}

\begin{figure}[htbp]
 \begin{center}
  \includegraphics[width=0.8\linewidth]{fig28.png}
 \end{center}
 \caption{330Hz,440Hz音叉}
 \label{fig:28}
\end{figure}

\begin{figure}[htbp]
 \begin{center}
  \includegraphics[width=0.8\linewidth]{fftmix.png}
 \end{center}
 \caption{信号のFFT}
 \label{fig:fftmix}
\end{figure}

\begin{figure}[htbp]
 \begin{center}
  \includegraphics[width=0.8\linewidth]{cfftmix.png}
 \end{center}
 \caption{フィルター後の関数}
 \label{fig:cfftmix}
\end{figure}

\noindent
\textbf{Task 5.2 ハイパスフィルター,ローパスフィルター} \\
取り込んだ音声は図\ref{fig:29}のようになった.
またF = 3000の時低い声がなくなり高い声のみがスピーカーで再生されるようになった.


\begin{figure}[htbp]
 \begin{center}
  \includegraphics[width=0.8\linewidth]{fig29.png}
 \end{center}
 \caption{取り込んだ声}
 \label{fig:29}
\end{figure}


\noindent
\textbf{Task 5.3 ボイスチェンジャー} \\
前回の実験同様サンプリング周波数をあげて先ほど取り込んだ音声を再生すると高い声に声を変化させることに成功した.
%============================================================
\subsection{Discussion}
\noindent
\textbf{Task 5.1 特定周波数の取り込み} \\
取り込んだ時間周波数領域の信号は周期的に振動していたので適切に測定することができたと予想できる.
またFFTしたのちのグラフにおいては330Hzと440Hzにツノが立っていることよりも正しく取り込めたと予想できる.

\noindent
\textbf{Task 5.2 ハイパスフィルター,ローパスフィルター} \\
サンプリング周波数20kHz,サンプリング数40000としたので1データあたり0.5Hz刻みになっていると予想できる.
したがって窓関数を作った際にF = 3000を境に低い声が聞こえなくなったということは低い声は1500Hz以下の声であったと考えられる.

\noindent
\textbf{Task 5.3 ボイスチェンジャー} \\
前回の実験で考察したので割愛する.

\noindent
\textbf{全体のまとめ} \\
今回は窓関数にステップ関数を用いたがこれはフーリエ逆変換する際に微分不可能点を生じさせることになり境界が滑らかにならないという欠点がある.そのため窓関数を正規分布のガウス関数を用いたりcos関数を用いるなどしてフィルターをかけた際の周波数領域関数を滑らかにすることでより正確な分離を行うことができると予想できる.

最後にフーリエ変換の応用例について考える.
今回はローパスフィルター,ハイパスフィルターを用いて音声をカットすることを考えたがこれは音声ファイルや画像ファイルにおいてデータ量を削減したいときに人間の目や耳では感知できないような高周波,低周波成分をカットすることによりデータ量を削減するなどの技術を考えることができる.
%=============================================================
\newpage
\end{document}

%=============================================================
\begin{thebibliography}{9}
  \bibitem{1} なっとくするフーリエ変換 小暮陽三
  \bibitem{2} ディジタル諡号処理 府川和彦
\end{thebibliography}
%=============================================================
\end{document}
